\documentclass{article}

\title{Riunione 1}
\date{2019-30-11}
\author{Mario Turco}

\begin{document}

\maketitle
\pagenumbering{gobble}
\newpage
\pagenumbering{arabic}

\section{Argomenti}

Prima discussione su come iniziare il progetto.
Dobbiamo definire i requisiti funzionali, non funzionali e di dominio (eventuali), definire gli attori e definire i casi d'uso.
Sul forum non c'è materiale rilevante che sia d'aiuto.

\section{Definizione degli attori}
  \paragraph{Lista di attori}
  \begin{enumerate}
      \item Amministratore
      \item Utente Registrato (cliente)
      \item Utente non Registrato (visitatore)
  \end{enumerate}
  \paragraph{Alcune considerazioni}
  A quanto pare la traccia fa una distinzione tra utenti non registrati , chiamati visitatori, ed utente registrati
  che, ad esempio possono lasciare recensioni a differenza dei visitatori.
  Per quanto riguarda gli amministratori, lato "Back Office", potrebbero ulteriormente dividersi in moderatori 
  delle recensioni che sono un sottoinsieme degli amministratori(i quali hanno tutte le funzioni dei moderatori)
 più qualcoshe altra funzione. Ma probabilmente non è una divisiore utile, ne discuteremo più avanti.

 \section{Lista di funzinalità per ogni attore}
\end{document}