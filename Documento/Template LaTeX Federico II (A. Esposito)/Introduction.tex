% ===========================================================================
%
%		FEDERICO II THESIS TEMPLATE - ENGLISH
%  					* an example of Introduction
%	 
% 		AUTHOR:  		Antonio Esposito (antonio.esposito103@studenti.unina.it)
%		LAST UPDATED:	2017/06/20
%
% ===========================================================================

\chapter*{Introduction}
\addcontentsline{toc}{chapter}{Introduction}

The introduction is prior to any chapter. 
It has to present the problem taken into exam and clarify the aim of the work, eventually pointing out the original contributions to the scientific literature; the introduction can cointain a brief summary of the thesis organization. The following chapters include the description of the state of the art concerning the discussed topic. For experimental theses, some chapters are dedicated to the description of the innovative solution and its validation. The conclusions, finally, must sum up briefly the faced challenges and report the main consideration that can be done after the experimental results.

The manuscript must include a bibliography that lists all the references (books, scientific articles, theses, web sites, etc.), together with proper information about where the reader can find the cited material. See the references of this file to understand the desired format for the citations.

In the following part of this file, more details on the preparation and presentation of the thesis are given in Chapter 1. Then, examples of typesetting of mathematical text are shown in Chapter 2. Table, figue, and software code are instead discussed in Chapter 3. Conclusions follow. The aim of the present file is to show an example of thesis structure, and, in its \LaTeX version, the main commands that can be useful during a thesis writing are used: in this way, the student that is willing to write his/her thesis with \LaTeX, has a good starting point. It's worth noting that the current version of the file was written by an italian student, and this means that lots of errors are present, expecially regarding the english grammar\dots

