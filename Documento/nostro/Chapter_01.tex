% ===========================================================================
%
%		FEDERICO II THESIS TEMPLATE - ENGLISH
%  					* an example of Chapter 1: information about the discussion of the thesis
%	 
% 		AUTHOR:  		Antonio Esposito (antonio.esposito103@studenti.unina.it)
%		LAST UPDATED:	2017/06/20
%
% ===========================================================================

\chapter{Modello funzionale}

This chapter contains useful information for the preparation and the presentation of the master degree thesis for students of Electronic Engineering (M61), at the University of Study of Naples Federico II.

The final test for the Master Degree course in Electronic Engineering consists in the preparation and discussion of a thesis, written with the help of a supervisor (eventually with one or two co-supervisors). This work is the final result of the student career and it testifies his/her ability in exploring in deep the topics encountered during the degree course.

%%%%% ===============================================================================
\section{Modellazione dei casi d'uso}

The supervisor is one of the professors that the candidate encountered during the degree course. Usually, the student finds its supervisor through informal talks, once provided that the professor is available and the student is interested in the professor's topics of interest. The degree course, on its website \url{www.ingegneria-elettronica.unina.it}, has defined a page with a non-esaustive list of available theses topics, in order to facilitate the information exchange between students and professors.

In case the thesis is developed after an intra-moenia internship, among one of the laboratories of the departement, the tutor that has already followed the student during the internship becomes the supervisor.

The supervisor defines the thesis topic. As already mentioned, the supervisor can be helped by one (or two maximum) co-supervisor. Supervisor and co-supervisor must guide and assist the student during the thesis developement and also provide him all the needed methodological and practical instruments. During the thesis are usually foreseen periodical meetings of the candidate with the supervisor, during which the ongoing work and the obtained results are discussed, also to define the future steps of the work.


%%%%% ===============================================================================
\section{Tabelle di Cockburn}
Di seguito si riportano, divisa per attori, le tabelle di Cockburn relative agli Use Case Diagram.
\subsection{Amministratore}

\begin{table}[h!]    
\def\arraystretch{1.5}
\caption{L'amministratore effettua il login }

%Intestazione tabella%
\begin{tabularx}{\textwidth}{|l|X|X|X|X|}
 \rowcolor{Gray}
  \hline Use Case \#1 & \multicolumn{4} {l|}{Effettua Login} \\ \hline Goal in
  Context & \multicolumn{4}{>{\hsize=\dimexpr 4\hsize+4\tabcolsep+2\arrayrulewidth\relax}X|}{%
    L'amministratore vuole accedere all'applicativo di Back-Office} \\
 \hline Preconditions & \multicolumn{4}{>{\hsize=\dimexpr 4\hsize+4\tabcolsep+2\arrayrulewidth\relax}X|}{%
 L'amministratore non ha effettuto lo use case "Effettua Login".  } \\
 \hline Success End Conditions &
 \multicolumn{4}{>{\hsize=\dimexpr 4\hsize+4\tabcolsep+2\arrayrulewidth\relax}X|}{ Il login dell'amministratore va a buon fine.} \\
 \hline Failed End Conditions &
 \multicolumn{4}{>{\hsize=\dimexpr 4\hsize+4\tabcolsep+2\arrayrulewidth\relax}X|}{I dati di login sono errati o il server non è raggiungibile.} \\
 \hline Primary Actor &
  \multicolumn{4}{l|}{Amministratore} \\
 \hline Trigger & 
 \multicolumn{4}{>{\hsize=\dimexpr 4\hsize+4\tabcolsep+2\arrayrulewidth\relax}X|}{L'amministratore preme il pulsante "Login" nella schermata LoginForm visibile all'avvio del software.} \\
\hline
%Main Scenario%
\multicolumn{5}{|>{\hsize=\dimexpr 4\hsize+4\tabcolsep+2\arrayrulewidth\relax}c|}{Main Scenario}\\\hline
\end{tabularx}
\setlength{\tabcolsep}{8pt}
\renewcommand{\arraystretch}{1.5}
    \begin{tabularx}{\textwidth}{|c|X|X|}
        Step\# & Amministratore & Sistema \\
        \hline
         1 &Compila correttamente i textField Username e Password  & \\
         \hline
         2 &Preme il pulsante "Login" dalla schermata LoginForm  & \\
         \hline
         3 &  &Mostra schermata HomePage\\
        \hline
    \end{tabularx} 
    \end{table}
    \begin{table}[H]
        \caption{Effettua Login - Estensione 1}
    %Extension 1%
    \begin{tabularx}{\textwidth}{|c|X|X|}
            \hline
            \rowcolor{LightGray}
            \multicolumn{3}{|>{\hsize=\dimexpr 4\hsize+4\tabcolsep+2\arrayrulewidth\relax}c|}{Extension 1: l'amministatore inserisce dati errati}\\\hline
            Step\# & Amministratore & Sistema \\
            \hline
             1 a &  Non complia o compila erroneamente i textField Username e Password& \\
             \hline
             2 a & Preme il pulsante Login & \\
             \hline
             3 a & & Mostra CredenzialiErrateDialog \\
             \hline
             4 a & Preme il tasto Ok&  \\
             \hline
             5 a & & Mostra LoginForm e termina caso d'uso\\
             \hline        
        \end{tabularx} 
    \end{table}
    \begin{table}[h!]
        \caption{Effettua Login - Estensione 2}
    %Extension 2%
    \begin{tabularx}{\textwidth}{|c|X|X|}
        \hline
        \rowcolor{LightGray}
        \multicolumn{3}{|>{\hsize=\dimexpr 4\hsize+4\tabcolsep+2\arrayrulewidth\relax}c|}{Extension 2: il server non è raggiungibile}\\\hline
        Step\# & Amministratore & Sistema \\
        \hline
         1 b &  Compila i campi username e password& \\
         \hline
         2 b & Preme il pulsante Login & \\
         \hline
         3 a & & Mostra 'Connessione assente dialog' e termina caso d'uso \\
         \hline
       
    \end{tabularx} 
\end{table}

\pagebreak

\begin{table}[H]    
\def\arraystretch{1.5}
\caption{L'amministratore valuta una recensione}

%Intestazione tabella%
\begin{tabularx}{\textwidth}{|l|X|X|X|X|}
  \rowcolor{Gray}
  \hline Use Case \#2 & \multicolumn{4} {l|}{Valuta Recensione} \\ \hline Goal in
  Context & \multicolumn{4}{>{\hsize=\dimexpr 4\hsize+4\tabcolsep+2\arrayrulewidth\relax}X|}{%
    L'amministratore valuta una recensione.} \\
 \hline Preconditions & \multicolumn{4}{>{\hsize=\dimexpr 4\hsize+4\tabcolsep+2\arrayrulewidth\relax}X|}{%
 L'amministratore ha effettuto lo use case "Effettua Login".  } \\
 \hline Success End Conditions &
 \multicolumn{4}{>{\hsize=\dimexpr 4\hsize+4\tabcolsep+2\arrayrulewidth\relax}X|}{ L'amministratore valuta una recensione. Il sistema tiene traccia di tale operazione.} \\
 \hline Failed End Conditions &
 \multicolumn{4}{>{\hsize=\dimexpr 4\hsize+4\tabcolsep+2\arrayrulewidth\relax}X|}{L'amministatore preme annulla. L'amministrazione valuta una recensione che è già stata valutata. Il server non è raggiungibile.} \\
 \hline Primary Actor &
  \multicolumn{4}{l|}{Amministratore} \\
 \hline Trigger & 
 \multicolumn{4}{>{\hsize=\dimexpr 4\hsize+4\tabcolsep+2\arrayrulewidth\relax}X|}{L'amministratore preme il pulsante "Recensioni" nella Homepage.} \\
\hline
%Main Scenario%
\multicolumn{5}{|>{\hsize=\dimexpr 4\hsize+4\tabcolsep+2\arrayrulewidth\relax}c|}{Main Scenario}\\\hline
\end{tabularx}
\setlength{\tabcolsep}{8pt}
\renewcommand{\arraystretch}{1.5}
    \begin{tabularx}{\textwidth}{|c|X|X|}
        Step\# & Amministratore & Sistema \\
        \hline
         1 &Preme il pulsante "Recensioni" nella schermata principale & \\
         \hline
         2 & & Mostra schermata "Recensioni"\\
         \hline
         3 & Clicca sulla card di una recensione  &\\
         \hline
         4 & & Mostra schermata "Visualizza Recensione"\\
       \hline
         5 & Clicca sul pulsante Accetta &\\
        \hline
        6& &Mostra 'Recensione Approvata Dialog' e termina lo use case\\
        \hline
    \end{tabularx}
    %Subvariation 1%
\end{table}
\begin{table}[h!]
    \caption{Valuta una recensione - Subvariation 1}
        \begin{tabularx}{\textwidth}{|c|X|X|}
            \hline
            \rowcolor{LightGray}
            \multicolumn{3}{|>{\hsize=\dimexpr 4\hsize+4\tabcolsep+2\arrayrulewidth\relax}c|}{Subvariation 1: l'amministatore rifiuta una recensione}\\\hline
            Step\# & Amministratore & Sistema \\
            \hline
             5 i &Preme il pulsante "Rifiuta". & \\
             \hline
             6 i & & Mostra 'Recensione Rifiutata Dialog' e termina lo use case.\\
            \hline
        \end{tabularx}
\setlength{\tabcolsep}{8pt}
\renewcommand{\arraystretch}{1.5}
\end{table}

 %Extension 1%
\begin{table}[h!]
    \caption{Valuta una recensione - Estensione 1}
    \begin{tabularx}{\textwidth}{|c|X|X|}
        \hline
        \rowcolor{LightGray}
        \multicolumn{3}{|>{\hsize=\dimexpr 4\hsize+4\tabcolsep+2\arrayrulewidth\relax}c|}{Extension 1: l'amministatore preme annulla}\\\hline
        Step\# & Amministratore & Sistema \\
        \hline
         3/5 a &Preme il pulsante "Recensioni" dal menu laterale sinistro & \\
         \hline
         4/6 a & & Ritorna alla schermata principale e termina il caso d'uso.\\
        \hline
    \end{tabularx}
\end{table}
%Estensione 2
\begin{table}[h!]
    \caption{Valuta una recensione - Estensione 2}
     \begin{tabularx}{\textwidth}{|c|X|X|}
        \hline
        \rowcolor{LightGray}
        \multicolumn{3}{|>{\hsize=\dimexpr 4\hsize+4\tabcolsep+2\arrayrulewidth\relax}c|}{Extension 2: la recensione è già stata valutata }\\\hline
         Step\# & Amministratore & Sistema \\
         \hline
          4 b  & & Mostra Fallimento Dialog e termina caso d'uso \\
          \hline
     \end{tabularx}
\end{table}
%Estensione 3
\begin{table}[h!]
    \caption{Valuta una recensione - Estensione 3}
     \begin{tabularx}{\textwidth}{|c|X|X|}
        \hline
        \rowcolor{LightGray}
        \multicolumn{3}{|>{\hsize=\dimexpr 4\hsize+4\tabcolsep+2\arrayrulewidth\relax}c|}{Extension 3: il server non è raggiungibile }\\\hline
         Step\# & Amministratore & Sistema \\
         \hline
          4/6 c  & & Mostra Connessione Assente Dialog e termina caso d'uso \\
          \hline
     \end{tabularx}
\end{table}
\pagebreak
\subsection{Utente non autenticato}

%Intestazione tabella%
\begin{table}
\begin{tabularx}{\textwidth}{|l|X|X|X|X|}
  \hline Use Case \#1 & \multicolumn{4} {l|}{Utente effettua Login} \\ \hline Goal in
  Context & \multicolumn{4}{>{\hsize=\dimexpr 4\hsize+4\tabcolsep+2\arrayrulewidth\relax}X|}{%
    L'utente non loggato effettua il login} \\
 \hline Preconditions & \multicolumn{4}{>{\hsize=\dimexpr 4\hsize+4\tabcolsep+2\arrayrulewidth\relax}X|}{%
 L'utente non autenticato non ha effettuto lo use case "Utente effettua Login".  } \\
 \hline Success End Conditions &
 \multicolumn{4}{>{\hsize=\dimexpr 4\hsize+4\tabcolsep+2\arrayrulewidth\relax}X|}{ Il login dell'utente va a buon fine.} \\
 \hline Failed End Conditions &
 \multicolumn{4}{>{\hsize=\dimexpr 4\hsize+4\tabcolsep+2\arrayrulewidth\relax}X|}{I dati di login sono errati oppure il server non è raggiungibile.} \\
 \hline Primary Actor &
  \multicolumn{4}{l|}{Utente non loggato} \\
 \hline Trigger & 
 \multicolumn{4}{>{\hsize=\dimexpr 4\hsize+4\tabcolsep+2\arrayrulewidth\relax}X|}{L'utente preme il pulsante 'Login' nel Navigation Drawer laterale dalla schermata 'HomePage utente non loggato'.} \\
\hline
%Main Scenario%
\multicolumn{5}{|>{\hsize=\dimexpr 4\hsize+4\tabcolsep+2\arrayrulewidth\relax}c|}{Main Scenario}\\\hline
\end{tabularx}
\setlength{\tabcolsep}{8pt}
\renewcommand{\arraystretch}{1.5}
    \begin{tabularx}{\textwidth}{|c|X|X|}
        Step\# & Utente & Sistema \\
        \hline
         1 &Compila correttamente i textField Username e Password  & \\
         \hline
         2 &Preme il pulsante "Login" dalla schermata LoginForm  & \\
         \hline
         3 &  &Mostra schermata HomePage\\
        \hline
    \end{tabularx}
    %Extension 1%
        \begin{tabularx}{\textwidth}{|c|X|X|}
            \hline
            \multicolumn{3}{|>{\hsize=\dimexpr 4\hsize+4\tabcolsep+2\arrayrulewidth\relax}c|}{Extension 1: l'utente inserisce dati errati}\\\hline
            Step\# & Utente & Sistema \\
            \hline
             1 a &  Compila erroneamente i textField Username e Password& \\
             \hline
             2 a & Preme il pulsante Login & \\
             \hline
             3 a & & Mostra schermata Login Dati Errati Dialog \\
             \hline
             4 a & Preme il tasto 'Riprova'&  \\
             \hline
             5 a & & Mostra Login e termina caso d'uso\\
             \hline        
        \end{tabularx} 
    %Extension 2%
    \begin{tabularx}{\textwidth}{|c|X|X|}
      \hline
      \multicolumn{3}{|>{\hsize=\dimexpr 4\hsize+4\tabcolsep+2\arrayrulewidth\relax}c|}{Extension 2: l'utente non compila tutti i campi}\\\hline
      Step\# & Utente & Sistema \\
      \hline
       1 b &  Non compila o compila solo un tra i textField Username e Password& \\
       \hline
       2 b & Preme il pulsante Login & \\
       \hline
       3 b & & Mostra schermata Login campi vuoti Dialog \\
       \hline
       4 b & Preme il tasto 'Riprova'&  \\
       \hline
       5 b & & Mostra Login e termina caso d'uso\\
       \hline        
  \end{tabularx}
\end{table}
\pagebreak
%Extension 3%
\begin{table}
\begin{tabularx}{\textwidth}{|c|X|X|}
  \hline
  \multicolumn{3}{|>{\hsize=\dimexpr 4\hsize+4\tabcolsep+2\arrayrulewidth\relax}c|}{Extension 3: il server risulta non raggiungibile}\\\hline
  Step\# & Utente & Sistema \\
  \hline
   3 c & & Mostra schermata Login server irraggiungibile \\
   \hline
   4 c & Preme il tasto 'Riprova'&  \\
   \hline
   5 c & & Mostra Login e termina caso d'uso\\
   \hline        
\end{tabularx} 
\end{table}

\pagebreak

%Intestazione tabella%
\begin{table}[h!]
\caption{L'utente non loggato effettua la registazione}
\begin{tabularx}{\textwidth}{|l|X|X|X|X|}
  \hline Use Case \#2 & \multicolumn{4} {l|}{L'utente non loggato si registra alla piattaforma} \\ \hline Goal in
  Context & \multicolumn{4}{>{\hsize=\dimexpr 4\hsize+4\tabcolsep+2\arrayrulewidth\relax}X|}{%
    L'utente non loggato effettua la registrazione} \\
 \hline Preconditions & \multicolumn{4}{>{\hsize=\dimexpr 4\hsize+4\tabcolsep+2\arrayrulewidth\relax}X|}{%
 -  } \\
 \hline Success End Conditions &
 \multicolumn{4}{>{\hsize=\dimexpr 4\hsize+4\tabcolsep+2\arrayrulewidth\relax}X|}{ La registrazione dell'utente va a buon fine.} \\
 \hline Failed End Conditions &
 \multicolumn{4}{>{\hsize=\dimexpr 4\hsize+4\tabcolsep+2\arrayrulewidth\relax}X|}{Il server non è raggiungibile o l'utente immette dati non validi.} \\
 \hline Primary Actor &
  \multicolumn{4}{l|}{Utente non loggato} \\
 \hline Trigger & 
 \multicolumn{4}{>{\hsize=\dimexpr 4\hsize+4\tabcolsep+2\arrayrulewidth\relax}X|}{L'utente preme il pulsante 'Registrati' nel Navigation Drawer laterale dalla schermata 'HomePage utente non loggato'.} \\
\hline
%Main Scenario%
\multicolumn{5}{|>{\hsize=\dimexpr 4\hsize+4\tabcolsep+2\arrayrulewidth\relax}c|}{Main Scenario}\\\hline
\end{tabularx}
\setlength{\tabcolsep}{8pt}
\renewcommand{\arraystretch}{1.5}
    \begin{tabularx}{\textwidth}{|c|X|X|}
        Step\# & Utente & Sistema \\
        \hline
         1 &Compila correttamente tutti i textField ed il date picker  & \\
         \hline
         2 &Preme il pulsante "Fine" dalla schermata Registrazione  & \\
         \hline
         3 &  &Mostra schermata "Registrazione success dialog" e termina il caso d'uso\\
        \hline
    \end{tabularx}
  \end{table}
  \begin{table}[h!]
    \caption{Effettua registrazione - Estensione 1}
    %Extension 1%
        \begin{tabularx}{\textwidth}{|c|X|X|}
            \hline
            \multicolumn{3}{|>{\hsize=\dimexpr 4\hsize+4\tabcolsep+2\arrayrulewidth\relax}c|}{Extension 1: l'utente inserisce dati di un utente già registrato}\\\hline
            Step\# & Utente & Sistema \\
            \hline
             1 a &  Compila i textField inserendo i dati di un account già registrato& \\
             \hline
             2 a & Preme il pulsante "Fine" & \\
             \hline
             3 a & & Mostra schermata "Registrazione utente esistente fail dialog" e termina il caso d'uso\\
             \hline        
        \end{tabularx} 
      \end{table}

    %Extension 2%
    \begin{table}[h!]
    \caption{Effettura registrazione - Estensione 2}
    \begin{tabularx}{\textwidth}{|c|X|X|}
      \hline
      \multicolumn{3}{|>{\hsize=\dimexpr 4\hsize+4\tabcolsep+2\arrayrulewidth\relax}c|}{Extension 2: l'utente non compila tutti i campi o li compila in modo errato}\\\hline
      Step\# & Utente & Sistema \\
      \hline
       1 b &  Non compila o compila erroneamente i textfiel ed il datepicker& \\
       \hline
       2 b & Preme il pulsante "Fine" & \\
       \hline
       3 b & & Mostra schermata "Campi non compilato o errati dialog" e termina caso d'uso  \\
       \hline        
  \end{tabularx}
\end{table}

%Extension 3%
\begin{table}[h!]
  \caption{Effettura registrazione - Estensione 3}
\begin{tabularx}{\textwidth}{|c|X|X|}
  \hline
  \multicolumn{3}{|>{\hsize=\dimexpr 4\hsize+4\tabcolsep+2\arrayrulewidth\relax}c|}{Extension 3: il server risulta non raggiungibile}\\\hline
  Step\# & Utente & Sistema \\
  \hline
  1 c & Compila correttamente tutti i campi della registrazione&  \\
  \hline
  2 c & Preme il tasto "Fine"&  \\
  \hline
  3 c & & Mostra schermata "Registrazione server irraggiungibile" e termina caso d'uso \\
   \hline
\end{tabularx} 
\end{table}

\pagebreak
\subsection{Utente autenticato}
%%%%% ===============================================================================
\section{Mockup}

The redaction of the thesis has to be carried on by the candidate indipendentely. A dissertation type thesis has the structure of a scientific article where it is required to derive, from the international literature, the most recent developments on the topic of interest, it is required to synthsize them, present them in an omogenous way, and finally compare the different approaches highlighting pros and cons of each of them. A sperimental type thesis has the structure of a scientific report, it faces a specific problem, typically within a more wide project of interest forthe supervisor, proposing a solution that is innovative if compared to the state of the art. A sperimental thesis also includes a validation of the proposed solution, made by means of experimental measuraments and/or numerical simulations.

%%%%% ===============================================================================
\section{Glossario}

During the thesis discussion, the candidate has at his/her disposal {\bfseries 12 minutes} for the final presentation. The 12 minutes limit is imperative and the committee chairman will take care of the observance of this limit. Thus, te candidate must pay attention in synthesizing in a proper way the done work.

For the final presentation, the candidate has to use a \emph{PowerPoint presentation}. For the time limit, the presentation must include a limited number of slides (more than 15 are not suggested!) and focus the attention on the main aspects of the thesis:
\begin{compactitem}
\item the faced problem
\item the state of the art
\item the adopted methodologies
\item the obtained results
\item other\dots
\end{compactitem}
highlighting, if it is the case, the personal contribution to the innovation. All the details are not essential and digressions are to avoid.

It is important that the presentaion is accurately organized and proved, and that the student expose its work in a clear way to the committee. At the end of the presentation, the committee could also ask clarifications or curiosities to the candidate.

%%%%% ===============================================================================

