% ===========================================================================
%
%		FEDERICO II THESIS TEMPLATE - ENGLISH
%  					* an example of Chapter 1: information about the discussion of the thesis
%	 
% 		AUTHOR:  		Antonio Esposito (antonio.esposito103@studenti.unina.it)
%		LAST UPDATED:	2017/06/20
%
% ===========================================================================

\chapter{Preparation and presentation of the thesis}

This chapter contains useful information for the preparation and the presentation of the master degree thesis for students of Electronic Engineering (M61), at the University of Study of Naples Federico II.

The final test for the Master Degree course in Electronic Engineering consists in the preparation and discussion of a thesis, written with the help of a supervisor (eventually with one or two co-supervisors). This work is the final result of the student career and it testifies his/her ability in exploring in deep the topics encountered during the degree course.

%%%%% ===============================================================================
\section{The supervisor and the topic}

The supervisor is one of the professors that the candidate encountered during the degree course. Usually, the student finds its supervisor through informal talks, once provided that the professor is available and the student is interested in the professor's topics of interest. The degree course, on its website \url{www.ingegneria-elettronica.unina.it}, has defined a page with a non-esaustive list of available theses topics, in order to facilitate the information exchange between students and professors.

In case the thesis is developed after an intra-moenia internship, among one of the laboratories of the departement, the tutor that has already followed the student during the internship becomes the supervisor.

The supervisor defines the thesis topic. As already mentioned, the supervisor can be helped by one (or two maximum) co-supervisor. Supervisor and co-supervisor must guide and assist the student during the thesis developement and also provide him all the needed methodological and practical instruments. During the thesis are usually foreseen periodical meetings of the candidate with the supervisor, during which the ongoing work and the obtained results are discussed, also to define the future steps of the work.


%%%%% ===============================================================================
\section{Thesis in a company}

Some theses can require part of the work to be done in a company. The degree course in Electronic Engineering promotes this kind of thesis, usually carried on as the culmination of an extra-moenia internship; in order to facilitate theses in a company, on the degree course website is also present a list of available intership available among companies in the electronic field.

\emph{For theses in a company, the supervisor from the University is necessarly complemented by a co-supervisor from the company.}

The topic of the thesis is to identify accordingly to both the supervisor from the University and from the company, and also according to the goals of the degree course training. The co-supervisor from the company, in addition to the duties mentioned before, must also follow the activities of the studient during the stay in the company, giving him/her the needed assistance. The company can ask to the student and the University supervisor to declare that some of the information and the material concerning the work is not to publish during the working period.

For the theses in company too, during the activities, are generally foreseen meetings between the candidate, the supervisor from the Univeristy and the co-supervisor from the company, during which the results and future steps are discussed.

%%%%% ===============================================================================
\section{Thesis writing}

The redaction of the thesis has to be carried on by the candidate indipendentely. A dissertation type thesis has the structure of a scientific article where it is required to derive, from the international literature, the most recent developments on the topic of interest, it is required to synthsize them, present them in an omogenous way, and finally compare the different approaches highlighting pros and cons of each of them. A sperimental type thesis has the structure of a scientific report, it faces a specific problem, typically within a more wide project of interest forthe supervisor, proposing a solution that is innovative if compared to the state of the art. A sperimental thesis also includes a validation of the proposed solution, made by means of experimental measuraments and/or numerical simulations.

\subsection{Structure of the manuscript}

The thesis can be prepared both in english or in italian and it is divided in chapters. This file offers an example for the structure of a thesis, as already pointed out in the introduction. The aim of the introductive chapter is already described in the introduction of this file, while examples of the bibliography references are reported later on. It's worth to say that the references must have a number and follow the style shown later, in the references part of this chapter \cite{example:book} \cite{example:article} \cite{example:web_docs}.

The thesis must be printed in triple hard copy for the final presentation. Its electronic version is exclusevely to send to the supervisor and co-supervisor, at least seven days before the discussion. The format to use for the title page (and the rest) is available on the web site of the course, both in the .docx format and in \LaTeX.

%%%%% ===============================================================================
\section{The final presentation}

During the thesis discussion, the candidate has at his/her disposal {\bfseries 12 minutes} for the final presentation. The 12 minutes limit is imperative and the committee chairman will take care of the observance of this limit. Thus, te candidate must pay attention in synthesizing in a proper way the done work.

For the final presentation, the candidate has to use a \emph{PowerPoint presentation}. For the time limit, the presentation must include a limited number of slides (more than 15 are not suggested!) and focus the attention on the main aspects of the thesis:
\begin{compactitem}
\item the faced problem
\item the state of the art
\item the adopted methodologies
\item the obtained results
\item other\dots
\end{compactitem}
highlighting, if it is the case, the personal contribution to the innovation. All the details are not essential and digressions are to avoid.

It is important that the presentaion is accurately organized and proved, and that the student expose its work in a clear way to the committee. At the end of the presentation, the committee could also ask clarifications or curiosities to the candidate.

%%%%% ===============================================================================
\section{The degree mark}

The committee attributes and records the final mark at the end of each graduation day. Even though the committee has full authority in the decision of the final mark, usually guide lines taking into account the student career are followed. For the formulation of the final mark, expessed out of 110, there are generally three contribution according to the relation:

\begin{equation}\label{eq01:final_mark}
V = A + B + C
\end{equation}

In the equation \ref{eq01:final_mark}, the \textbf{A} part is related to the average score calculated relying on the exams the student has sustained during the career, according to the relation $A = (11/3)\times M$ where \textbf{M} is the average score out of 30, calculating by weighing the mark of each exam with the relative CFU (the \textbf{A} is properly increased if the result of some exams is 30 cum laude).

The \textbf{B}, that is $B\leqslant 3$, is related to the student career and also takes into account the final mark of the bachelor degree.

The \textbf{C}, that is $C\leqslant 5$, finally, is assigned by the committee according to the final exam and thesis presentation.

%%%%% ===============================================================================
\section{The burocracy}

Burocratic aspects regarding the thesis are in the domain of the students secretariat, at hich one should refer for any doubt, information or specific question. The student shall be required to be informed of the expiring dates for the delivering of all the necessary documents in advance.