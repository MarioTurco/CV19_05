% ===========================================================================
%
%		FEDERICO II THESIS TEMPLATE - ENGLISH
%  					* an example of Chapter 1: information about the discussion of the thesis
%	 
% 		AUTHOR:  		Antonio Esposito (antonio.esposito103@studenti.unina.it)
%		LAST UPDATED:	2017/06/20
%
% ===========================================================================
\chapter{Introduzione}
Lo scopo di questo documento è quello di specificare i requisiti del sistema software "Cerca Viaggi" per facilitarne la realizzazione e la validazione ed, in particolare,
si vogliono formalizzare i requisiti funzionali e non funzionali del sistema. \\Il documento prevede vari livelli di raffinamento partendo dal linguaggio naturale
ed arrivando ad un linguaggio strutturato ed a modelli UML.
\section{Requisiti Funzionali}
\begin{enumerate}
    \item  \textbf{Accesso amministratore al sistema}: un amministratore dovrà poter accedere al sistema tramite email e password (usando l'applicativo dekstop).
    \item  \textbf{Amministratore valuta una recensione}: un amministatore dovrà poter accettare o rifiutare una recensione scritta da un utente decidendo quindi se questa verrà pubblicata o eliminata.
    \item \textbf{Accesso utente al sistema}: un utente, dopo aver effettuato la registrazione, dovrà poter accedere al suo account tramite email e password (usando l'applicativo mobile). 
    \item  \textbf{Registrazione di un nuovo utente}: l'applicativo mobile dovrà permettere ad un utente, non precedentemente registrato, di creare un account specificando 
    i propri dati anagrafici (nome, cognome e data di nascita), i dati per effettuare l'accesso (email e passowrd) ed un nickname da utilizzare, secondo volonta dell'utente, per firmare  
    le proprie recensioni
    \item \textbf{Amministratore modifica dati dei visitatori}: un amministratore dovrà poter modificare i dati di un utente in particolare, email e password
    \item \textbf{Amministratore elimina account utente}: un amministratore dovrà poter eliminare l'account di un utente
    \item \textbf{Amministratore visualizza dati di un visitatore}: un amministratore dovrà poter visualizzare i dati anagrafici di un visitatore (nome, cognome e data di nascita), 
    il nickname ed il numero di recensioni approvate e rifiutate.
    \item \textbf{Utente visualizza struttra}: un utente dovrà poter visualizzare i dettagli di una struttura
    \item \textbf{Utente aggiunge una recensione}: un utente, una volta aver effettuato l'accesso, dovrà poter aggiungere una recensione ad una struttura
    \item \textbf{Utente visualizaza dettagli recensione}: un utente dovrà poter leggere i dettagli di una recensione, in particolare leggendone l'autore, la data di inserimento, la valutazione
    ed il testo integrale.
\end{enumerate}
\section{Requisiti non funzionali}
\begin{enumerate}
    \item Il sistema deve garantire una buona usabilità per tutti coloro che ne usufruiscono.
\end{enumerate}
\chapter{Modello funzionale}

This chapter contains useful information for the preparation and the presentation of the master degree thesis for students of Electronic Engineering (M61), at the University of Study of Naples Federico II.

The final test for the Master Degree course in Electronic Engineering consists in the preparation and discussion of a thesis, written with the help of a supervisor (eventually with one or two co-supervisors). This work is the final result of the student career and it testifies his/her ability in exploring in deep the topics encountered during the degree course.

%%%%% ===============================================================================
\section{Modellazione dei casi d'uso}

The supervisor is one of the professors that the candidate encountered during the degree course. Usually, the student finds its supervisor through informal talks, once provided that the professor is available and the student is interested in the professor's topics of interest. The degree course, on its website \url{www.ingegneria-elettronica.unina.it}, has defined a page with a non-esaustive list of available theses topics, in order to facilitate the information exchange between students and professors.

In case the thesis is developed after an intra-moenia internship, among one of the laboratories of the departement, the tutor that has already followed the student during the internship becomes the supervisor.

The supervisor defines the thesis topic. As already mentioned, the supervisor can be helped by one (or two maximum) co-supervisor. Supervisor and co-supervisor must guide and assist the student during the thesis developement and also provide him all the needed methodological and practical instruments. During the thesis are usually foreseen periodical meetings of the candidate with the supervisor, during which the ongoing work and the obtained results are discussed, also to define the future steps of the work.


%%%%% ===============================================================================
\section{Tabelle di Cockburn}
Di seguito si riportano, divise per attori, le tabelle di Cockburn relative agli Use Case Diagram.
\subsection{Amministratore}

\begin{table}[h!]    
\def\arraystretch{1.5}
\caption{L'amministratore effettua il login }

%Intestazione tabella%
\begin{tabularx}{\textwidth}{|l|X|X|X|X|}
 \rowcolor{Gray}
  \hline Use Case \#1 & \multicolumn{4} {l|}{Effettua Login} \\ \hline Goal in
  Context & \multicolumn{4}{>{\hsize=\dimexpr 4\hsize+4\tabcolsep+2\arrayrulewidth\relax}X|}{%
    L'amministratore vuole accedere all'applicativo di Back-Office} \\
 \hline Preconditions & \multicolumn{4}{>{\hsize=\dimexpr 4\hsize+4\tabcolsep+2\arrayrulewidth\relax}X|}{%
 L'amministratore non ha effettuto lo use case "Effettua Login".  } \\
 \hline Success End Conditions &
 \multicolumn{4}{>{\hsize=\dimexpr 4\hsize+4\tabcolsep+2\arrayrulewidth\relax}X|}{ Il login dell'amministratore va a buon fine.} \\
 \hline Failed End Conditions &
 \multicolumn{4}{>{\hsize=\dimexpr 4\hsize+4\tabcolsep+2\arrayrulewidth\relax}X|}{I dati di login sono errati o il server non è raggiungibile.} \\
 \hline Primary Actor &
  \multicolumn{4}{l|}{Amministratore} \\
 \hline Trigger & 
 \multicolumn{4}{>{\hsize=\dimexpr 4\hsize+4\tabcolsep+2\arrayrulewidth\relax}X|}{L'amministratore preme il pulsante "Login" nella schermata LoginForm visibile all'avvio del software.} \\
\hline
%Main Scenario%
\multicolumn{5}{|>{\hsize=\dimexpr 4\hsize+4\tabcolsep+2\arrayrulewidth\relax}c|}{Main Scenario}\\\hline
\end{tabularx}
\setlength{\tabcolsep}{8pt}
\renewcommand{\arraystretch}{1.5}
    \begin{tabularx}{\textwidth}{|c|X|X|}
        Step\# & Amministratore & Sistema \\
        \hline
         1 &Compila correttamente i textField Username e Password  & \\
         \hline
         2 &Preme il pulsante "Login" dalla schermata LoginForm  & \\
         \hline
         3 &  &Mostra schermata HomePage\\
        \hline
    \end{tabularx} 
    \end{table}
    \begin{table}[H]
        \caption{Effettua Login - Estensione 1}
    %Extension 1%
    \begin{tabularx}{\textwidth}{|c|X|X|}
            \hline
            \rowcolor{LightGray}
            \multicolumn{3}{|>{\hsize=\dimexpr 4\hsize+4\tabcolsep+2\arrayrulewidth\relax}c|}{Extension 1: l'amministatore inserisce dati errati}\\\hline
            Step\# & Amministratore & Sistema \\
            \hline
             1 a &  Non complia o compila erroneamente i textField Username e Password& \\
             \hline
             2 a & Preme il pulsante Login & \\
             \hline
             3 a & & Mostra CredenzialiErrateDialog e termina caso d'uso \\
             \hline      
        \end{tabularx} 
    \end{table}
    \begin{table}[h!]
        \caption{Effettua Login - Estensione 2}
    %Extension 2%
    \begin{tabularx}{\textwidth}{|c|X|X|}
        \hline
        \rowcolor{LightGray}
        \multicolumn{3}{|>{\hsize=\dimexpr 4\hsize+4\tabcolsep+2\arrayrulewidth\relax}c|}{Extension 2: il server non è raggiungibile}\\\hline
        Step\# & Amministratore & Sistema \\
        \hline
         1 b &  Compila i campi username e password& \\
         \hline
         2 b & Preme il pulsante Login & \\
         \hline
         3 a & & Mostra 'Connessione assente dialog' e termina caso d'uso \\
         \hline
       
    \end{tabularx} 
\end{table}

\documentclass[a4paper]{article}
\usepackage[T1]{fontenc}
\usepackage[utf8]{inputenc}
\usepackage{geometry}
\usepackage{multirow}
\usepackage{float} 
\usepackage{tabularx}


\geometry{
  left=30mm,
  right=30mm
}

\begin{document}
\begin{table}[H]    
\def\arraystretch{1.5}


%Intestazione tabella%
\begin{tabularx}{\textwidth}{|l|X|X|X|X|}
  \hline Use Case \#2 & \multicolumn{4} {l|}{Valuta Recensione} \\ \hline Goal in
  Context & \multicolumn{4}{>{\hsize=\dimexpr 4\hsize+4\tabcolsep+2\arrayrulewidth\relax}X|}{%
    L'amministratore valuta una recensione.} \\
 \hline Preconditions & \multicolumn{4}{>{\hsize=\dimexpr 4\hsize+4\tabcolsep+2\arrayrulewidth\relax}X|}{%
 L'amministratore ha effettuto lo use case "Effettua Login".  } \\
 \hline Success End Conditions &
 \multicolumn{4}{>{\hsize=\dimexpr 4\hsize+4\tabcolsep+2\arrayrulewidth\relax}X|}{ L'amministratore valuta una recensione. Il sistema tiene traccia di tale operazione.} \\
 \hline Failed End Conditions &
 \multicolumn{4}{>{\hsize=\dimexpr 4\hsize+4\tabcolsep+2\arrayrulewidth\relax}X|}{L'amministatore preme annulla. L'amministrazione valuta una recensione che è già stata valutata.} \\
 \hline Primary Actor &
  \multicolumn{4}{l|}{Amministratore} \\
 \hline Trigger & 
 \multicolumn{4}{>{\hsize=\dimexpr 4\hsize+4\tabcolsep+2\arrayrulewidth\relax}X|}{L'amministratore preme il pulsante "Recensioni" nella Homepage.} \\
\hline
%Main Scenario%
\multicolumn{5}{|>{\hsize=\dimexpr 4\hsize+4\tabcolsep+2\arrayrulewidth\relax}c|}{Main Scenario}\\\hline
\end{tabularx}
\setlength{\tabcolsep}{8pt}
\renewcommand{\arraystretch}{1.5}
    \begin{tabularx}{\textwidth}{|c|X|X|}
        Step\# & Amministratore & Sistema \\
        \hline
         1 &Preme il pulsante "Recensioni" nella schermata principale & \\
         \hline
         2 & & Mostra GestioneRecensioni\\
         \hline
         3 & Clicca sul radio button accanto ad una recensione e preme il pulsante "Conferma" &\\
         \hline
         4 & & Mostra ValutaRecensione\\
       \hline
         5 & Clicca sul pulsante Approva &\\
        \hline
        6& &Mostra RecensioneApprovataDialog\\
        \hline
    \end{tabularx}
    %Extension 1%
        \begin{tabularx}{\textwidth}{|c|X|X|}
            \hline
            \multicolumn{3}{|>{\hsize=\dimexpr 4\hsize+4\tabcolsep+2\arrayrulewidth\relax}c|}{Extension 1: l'amministatore rifiuta una recensione}\\\hline
            Step\# & Amministratore & Sistema \\
            \hline
             5 a &Preme il pulsante "Rifiuta". & \\
             \hline
             6 a & & Mostra RecensioneEliminataDialog.\\
            \hline
        \end{tabularx}
    
    %Extension 1%

\setlength{\tabcolsep}{8pt}
\renewcommand{\arraystretch}{1.5}
    \begin{tabularx}{\textwidth}{|c|X|X|}
        \hline
        \multicolumn{3}{|>{\hsize=\dimexpr 4\hsize+4\tabcolsep+2\arrayrulewidth\relax}c|}{Extension 2: l'amministatore preme annulla}\\\hline
        Step\# & Amministratore & Sistema \\
        \hline
         3/5 b &Preme il pulsante "Annulla". & \\
         \hline
         4/6 b & & Ritorna alla schermata principale e termina il caso d'uso.\\
        \hline
    \end{tabularx}

%Extension 2%
     \begin{tabularx}{\textwidth}{|c|X|X|}
        \hline
        \multicolumn{3}{|>{\hsize=\dimexpr 4\hsize+4\tabcolsep+2\arrayrulewidth\relax}c|}{Extension 3: la recensione è già stata valutata }\\\hline
         Step\# & Amministratore & Sistema \\
         \hline
          6 c  & & Mostra Fallimento Dialog \\
          \hline
          7 c & Preme Ok & \\
          \hline
          8 c &  & Ritorna alla schermata principale e termina il caso d'uso.\\
         \hline
     \end{tabularx}
\end{table}
\paragraph{Nota 1} Dato che il sistema può essere gestito da più amministratori è possibile che due di questi aprano contemporaneamente
la schermata di valutazione della stessa Recensione. In questo caso andrà a buon fine soltato la valutazione dell'amministatore
che per primo la rifiuterà o confermerà. L'altro amministratore vedrà la schermata di errore come descritto nella Extension 2. 
\end{document}
\input{Cockburn/Amministratore/Visualizza dati dei visitatori/visualizzaDati.tex}
\pagebreak
\subsection{Utente non autenticato} %finito

%Intestazione tabella%
\begin{table}
\begin{tabularx}{\textwidth}{|l|X|X|X|X|}
  \hline Use Case \#1 & \multicolumn{4} {l|}{Utente effettua Login} \\ \hline Goal in
  Context & \multicolumn{4}{>{\hsize=\dimexpr 4\hsize+4\tabcolsep+2\arrayrulewidth\relax}X|}{%
    L'utente non loggato effettua il login} \\
 \hline Preconditions & \multicolumn{4}{>{\hsize=\dimexpr 4\hsize+4\tabcolsep+2\arrayrulewidth\relax}X|}{%
 L'utente non autenticato non ha effettuto lo use case "Utente effettua Login".  } \\
 \hline Success End Conditions &
 \multicolumn{4}{>{\hsize=\dimexpr 4\hsize+4\tabcolsep+2\arrayrulewidth\relax}X|}{ Il login dell'utente va a buon fine.} \\
 \hline Failed End Conditions &
 \multicolumn{4}{>{\hsize=\dimexpr 4\hsize+4\tabcolsep+2\arrayrulewidth\relax}X|}{I dati di login sono errati oppure il server non è raggiungibile.} \\
 \hline Primary Actor &
  \multicolumn{4}{l|}{Utente non loggato} \\
 \hline Trigger & 
 \multicolumn{4}{>{\hsize=\dimexpr 4\hsize+4\tabcolsep+2\arrayrulewidth\relax}X|}{L'utente preme il pulsante 'Login' nel Navigation Drawer laterale dalla schermata 'HomePage utente non loggato'.} \\
\hline
%Main Scenario%
\multicolumn{5}{|>{\hsize=\dimexpr 4\hsize+4\tabcolsep+2\arrayrulewidth\relax}c|}{Main Scenario}\\\hline
\end{tabularx}
\setlength{\tabcolsep}{8pt}
\renewcommand{\arraystretch}{1.5}
    \begin{tabularx}{\textwidth}{|c|X|X|}
        Step\# & Utente & Sistema \\
        \hline
         1 &Compila correttamente i textField Username e Password  & \\
         \hline
         2 &Preme il pulsante "Login" dalla schermata LoginForm  & \\
         \hline
         3 &  &Mostra schermata HomePage\\
        \hline
    \end{tabularx}
    %Extension 1%
        \begin{tabularx}{\textwidth}{|c|X|X|}
            \hline
            \multicolumn{3}{|>{\hsize=\dimexpr 4\hsize+4\tabcolsep+2\arrayrulewidth\relax}c|}{Extension 1: l'utente inserisce dati errati}\\\hline
            Step\# & Utente & Sistema \\
            \hline
             1 a &  Compila erroneamente i textField Username e Password& \\
             \hline
             2 a & Preme il pulsante Login & \\
             \hline
             3 a & & Mostra schermata Login Dati Errati Dialog \\
             \hline
             4 a & Preme il tasto 'Riprova'&  \\
             \hline
             5 a & & Mostra Login e termina caso d'uso\\
             \hline        
        \end{tabularx} 
    %Extension 2%
    \begin{tabularx}{\textwidth}{|c|X|X|}
      \hline
      \multicolumn{3}{|>{\hsize=\dimexpr 4\hsize+4\tabcolsep+2\arrayrulewidth\relax}c|}{Extension 2: l'utente non compila tutti i campi}\\\hline
      Step\# & Utente & Sistema \\
      \hline
       1 b &  Non compila o compila solo un tra i textField Username e Password& \\
       \hline
       2 b & Preme il pulsante Login & \\
       \hline
       3 b & & Mostra schermata Login campi vuoti Dialog \\
       \hline
       4 b & Preme il tasto 'Riprova'&  \\
       \hline
       5 b & & Mostra Login e termina caso d'uso\\
       \hline        
  \end{tabularx}
\end{table}
\pagebreak
%Extension 3%
\begin{table}
\begin{tabularx}{\textwidth}{|c|X|X|}
  \hline
  \multicolumn{3}{|>{\hsize=\dimexpr 4\hsize+4\tabcolsep+2\arrayrulewidth\relax}c|}{Extension 3: il server risulta non raggiungibile}\\\hline
  Step\# & Utente & Sistema \\
  \hline
   3 c & & Mostra schermata Login server irraggiungibile \\
   \hline
   4 c & Preme il tasto 'Riprova'&  \\
   \hline
   5 c & & Mostra Login e termina caso d'uso\\
   \hline        
\end{tabularx} 
\end{table}


%Intestazione tabella%
\begin{table}[h!]
\caption{L'utente non loggato effettua la registazione}
\begin{tabularx}{\textwidth}{|l|X|X|X|X|}
  \hline Use Case \#2 & \multicolumn{4} {l|}{L'utente non loggato si registra alla piattaforma} \\ \hline Goal in
  Context & \multicolumn{4}{>{\hsize=\dimexpr 4\hsize+4\tabcolsep+2\arrayrulewidth\relax}X|}{%
    L'utente non loggato effettua la registrazione} \\
 \hline Preconditions & \multicolumn{4}{>{\hsize=\dimexpr 4\hsize+4\tabcolsep+2\arrayrulewidth\relax}X|}{%
 -  } \\
 \hline Success End Conditions &
 \multicolumn{4}{>{\hsize=\dimexpr 4\hsize+4\tabcolsep+2\arrayrulewidth\relax}X|}{ La registrazione dell'utente va a buon fine.} \\
 \hline Failed End Conditions &
 \multicolumn{4}{>{\hsize=\dimexpr 4\hsize+4\tabcolsep+2\arrayrulewidth\relax}X|}{Il server non è raggiungibile o l'utente immette dati non validi.} \\
 \hline Primary Actor &
  \multicolumn{4}{l|}{Utente non loggato} \\
 \hline Trigger & 
 \multicolumn{4}{>{\hsize=\dimexpr 4\hsize+4\tabcolsep+2\arrayrulewidth\relax}X|}{L'utente preme il pulsante 'Registrati' nel Navigation Drawer laterale dalla schermata 'HomePage utente non loggato'.} \\
\hline
%Main Scenario%
\multicolumn{5}{|>{\hsize=\dimexpr 4\hsize+4\tabcolsep+2\arrayrulewidth\relax}c|}{Main Scenario}\\\hline
\end{tabularx}
\setlength{\tabcolsep}{8pt}
\renewcommand{\arraystretch}{1.5}
    \begin{tabularx}{\textwidth}{|c|X|X|}
        Step\# & Utente & Sistema \\
        \hline
         1 &Compila correttamente tutti i textField ed il date picker  & \\
         \hline
         2 &Preme il pulsante "Fine" dalla schermata Registrazione  & \\
         \hline
         3 &  &Mostra schermata "Registrazione success dialog" e termina il caso d'uso\\
        \hline
    \end{tabularx}
  \end{table}
  \begin{table}[h!]
    \caption{Effettua registrazione - Estensione 1}
    %Extension 1%
        \begin{tabularx}{\textwidth}{|c|X|X|}
            \hline
            \multicolumn{3}{|>{\hsize=\dimexpr 4\hsize+4\tabcolsep+2\arrayrulewidth\relax}c|}{Extension 1: l'utente inserisce dati di un utente già registrato}\\\hline
            Step\# & Utente & Sistema \\
            \hline
             1 a &  Compila i textField inserendo i dati di un account già registrato& \\
             \hline
             2 a & Preme il pulsante "Fine" & \\
             \hline
             3 a & & Mostra schermata "Registrazione già registrato fail dialog e termina il caso d'uso\\
             \hline        
        \end{tabularx} 
      \end{table}

    %Extension 2%
    \begin{table}[h!]
    \caption{Effettura registrazione - Estensione 2}
    \begin{tabularx}{\textwidth}{|c|X|X|}
      \hline
      \multicolumn{3}{|>{\hsize=\dimexpr 4\hsize+4\tabcolsep+2\arrayrulewidth\relax}c|}{Extension 2: l'utente non compila tutti i campi o li compila in modo errato}\\\hline
      Step\# & Utente & Sistema \\
      \hline
       1 b &  Non compila o compila erroneamente i textfiel ed il datepicker& \\
       \hline
       2 b & Preme il pulsante "Fine" & \\
       \hline
       3 b & & Mostra schermata Login campi vuoti Dialog \\
       \hline
       4 b & Preme il tasto 'Riprova'&  \\
       \hline
       5 b & & Mostra Login e termina caso d'uso\\
       \hline        
  \end{tabularx}
\end{table}
\pagebreak
%Extension 3%
\begin{table}[h!]
  \caption{Effettura registrazione - Estensione 3}
\begin{tabularx}{\textwidth}{|c|X|X|}
  \hline
  \multicolumn{3}{|>{\hsize=\dimexpr 4\hsize+4\tabcolsep+2\arrayrulewidth\relax}c|}{Extension 3: il server risulta non raggiungibile}\\\hline
  Step\# & Utente & Sistema \\
  \hline
   3 c & & Mostra schermata Login server irraggiungibile \\
   \hline
   4 c & Preme il tasto 'Riprova'&  \\
   \hline
   5 c & & Mostra Login e termina caso d'uso\\
   \hline        
\end{tabularx} 
\end{table}


%Intestazione tabella%
\begin{table}[H]
    \caption{L'utente non loggato visulizza una struttura}
    \begin{tabularx}{\textwidth}{|l|X|X|X|X|}
      \hline Use Case \#3 & \multicolumn{4} {l|}{L'utente non loggato visulizza una struttura} \\ \hline Goal in
      Context & \multicolumn{4}{>{\hsize=\dimexpr 4\hsize+4\tabcolsep+2\arrayrulewidth\relax}X|}{%
        L'utente non loggato visulizza una struttura} \\
     \hline Preconditions & \multicolumn{4}{>{\hsize=\dimexpr 4\hsize+4\tabcolsep+2\arrayrulewidth\relax}X|}{%
     L'Utente ha effettuato una ricerca e si trova nella schermata "Lista Strutture" } \\
     \hline Success End Conditions &
     \multicolumn{4}{>{\hsize=\dimexpr 4\hsize+4\tabcolsep+2\arrayrulewidth\relax}X|}{ L'utente visualizza i dettagli di una struttura} \\
     \hline Failed End Conditions &
     \multicolumn{4}{>{\hsize=\dimexpr 4\hsize+4\tabcolsep+2\arrayrulewidth\relax}X|}{Il server non è raggiungibile } \\
     \hline Primary Actor &
      \multicolumn{4}{l|}{Utente non loggato} \\
     \hline Trigger & 
     \multicolumn{4}{>{\hsize=\dimexpr 4\hsize+4\tabcolsep+2\arrayrulewidth\relax}X|}{} \\
    \hline
    %Main Scenario%
    \multicolumn{5}{|>{\hsize=\dimexpr 4\hsize+4\tabcolsep+2\arrayrulewidth\relax}c|}{Main Scenario}\\\hline
    \end{tabularx}
    \setlength{\tabcolsep}{8pt}
    \renewcommand{\arraystretch}{1.5}
        \begin{tabularx}{\textwidth}{|c|X|X|}
            Step\# & Utente & Sistema \\
            \hline
             1 & L'utente clicca la Card di una struttura dalla schermata "Lista Strutture"'. & \\
             \hline
             2 && Mostra la schermata "Pagina Struttura utente non loggato" e termina caso d'uso\\
             \hline
            
        \end{tabularx}
    \end{table}
    \begin{table}[H]
    \caption{Visualizza struttura - Estensione 1}
    %Extension 1%
         \begin{tabularx}{\textwidth}{|c|X|X|}
                \hline
                \multicolumn{3}{|>{\hsize=\dimexpr 4\hsize+4\tabcolsep+2\arrayrulewidth\relax}c|}{Extension 1: il server non è raggiungibile}\\\hline
                Step\# & Utente & Sistema \\
                \hline
                 2 a &  & Mostra schermata "Connessione assente" e termina caso d'uso\\
                 \hline 
        \end{tabularx} 
\end{table}
    
       


%Intestazione tabella%
\begin{table}[H]
    \caption{L'utente  visulizza i dettagli di una recensione}
    \begin{tabularx}{\textwidth}{|l|X|X|X|X|}
      \hline 
      \rowcolor{Gray}
      Use Case \#4 & \multicolumn{4} {l|}{L'utente  visulizza i dettagli di una recensione} \\ \hline Goal in
      Context & \multicolumn{4}{>{\hsize=\dimexpr 4\hsize+4\tabcolsep+2\arrayrulewidth\relax}X|}
      {L'utente vuole visualizzare il testo completo di una recensione} \\
     \hline Preconditions & \multicolumn{4}{>{\hsize=\dimexpr 4\hsize+4\tabcolsep+2\arrayrulewidth\relax}X|}{%
      } \\
     \hline Success End Conditions &
     \multicolumn{4}{>{\hsize=\dimexpr 4\hsize+4\tabcolsep+2\arrayrulewidth\relax}X|}{ L'utente visualizza i dettagli di una recensione} \\
     \hline Failed End Conditions &
     \multicolumn{4}{>{\hsize=\dimexpr 4\hsize+4\tabcolsep+2\arrayrulewidth\relax}X|}{Il server non è raggiungibile } \\
     \hline Primary Actor &
      \multicolumn{4}{X|}{Utente} \\
     \hline Trigger & 
     \multicolumn{4}{>{\hsize=\dimexpr 4\hsize+4\tabcolsep+2\arrayrulewidth\relax}X|}{L'utente clicca sulla card di una recensione dalla schermata "Pagina Struttura"} \\
    \hline
    %Main Scenario%
    \multicolumn{5}{|>{\hsize=\dimexpr 4\hsize+4\tabcolsep+2\arrayrulewidth\relax}c|}{Main Scenario}\\
    \hline
\end{tabularx}
    \setlength{\tabcolsep}{8pt}
    \renewcommand{\arraystretch}{1.5}
        \begin{tabularx}{\textwidth}{|c|X|X|}
            
            Step\# & Utente & Sistema \\
            \hline
             1 & Clicca la Card di una recensione dalla schermata "Pagina struttura"'. & \\
             \hline
             2 & & Mostra la schermata "Pagina Recensione" e termina caso d'uso\\
             \hline
        \end{tabularx}
   \end{table}
    \begin{table}[H]
    \caption{Visualizza struttura - Estensione 1}
    %Extension 1%
    
         \begin{tabularx}{\textwidth}{|c|X|X|}
                \hline
                \rowcolor{LightGray}
                \multicolumn{3}{|>{\hsize=\dimexpr 4\hsize+4\tabcolsep+2\arrayrulewidth\relax}c|}{Extension 1: il server non è raggiungibile}\\\hline
                Step\# & Utente & Sistema \\
                \hline
                 2 a &  & Mostra schermata "Connessione assente" e termina caso d'uso\\
                 \hline 
        \end{tabularx} 
\end{table}
    
       


\subsection{Utente autenticato}

%Intestazione tabella%
\begin{table}[H]
    \caption{L'utente non loggato visulizza una struttura}
    \begin{tabularx}{\textwidth}{|l|X|X|X|X|}
      \hline Use Case \#3 & \multicolumn{4} {l|}{L'utente non loggato visulizza una struttura} \\ \hline Goal in
      Context & \multicolumn{4}{>{\hsize=\dimexpr 4\hsize+4\tabcolsep+2\arrayrulewidth\relax}X|}{%
        L'utente non loggato visulizza una struttura} \\
     \hline Preconditions & \multicolumn{4}{>{\hsize=\dimexpr 4\hsize+4\tabcolsep+2\arrayrulewidth\relax}X|}{%
     L'Utente ha effettuato una ricerca e si trova nella schermata "Lista Strutture" } \\
     \hline Success End Conditions &
     \multicolumn{4}{>{\hsize=\dimexpr 4\hsize+4\tabcolsep+2\arrayrulewidth\relax}X|}{ L'utente visualizza i dettagli di una struttura} \\
     \hline Failed End Conditions &
     \multicolumn{4}{>{\hsize=\dimexpr 4\hsize+4\tabcolsep+2\arrayrulewidth\relax}X|}{Il server non è raggiungibile } \\
     \hline Primary Actor &
      \multicolumn{4}{l|}{Utente non loggato} \\
     \hline Trigger & 
     \multicolumn{4}{>{\hsize=\dimexpr 4\hsize+4\tabcolsep+2\arrayrulewidth\relax}X|}{} \\
    \hline
    %Main Scenario%
    \multicolumn{5}{|>{\hsize=\dimexpr 4\hsize+4\tabcolsep+2\arrayrulewidth\relax}c|}{Main Scenario}\\\hline
    \end{tabularx}
    \setlength{\tabcolsep}{8pt}
    \renewcommand{\arraystretch}{1.5}
        \begin{tabularx}{\textwidth}{|c|X|X|}
            Step\# & Utente & Sistema \\
            \hline
             1 & L'utente clicca la Card di una struttura dalla schermata "Lista Strutture"'. & \\
             \hline
             2 && Mostra la schermata "Pagina Struttura utente non loggato" e termina caso d'uso\\
             \hline
            
        \end{tabularx}
    \end{table}
    \begin{table}[H]
    \caption{Visualizza struttura - Estensione 1}
    %Extension 1%
         \begin{tabularx}{\textwidth}{|c|X|X|}
                \hline
                \multicolumn{3}{|>{\hsize=\dimexpr 4\hsize+4\tabcolsep+2\arrayrulewidth\relax}c|}{Extension 1: il server non è raggiungibile}\\\hline
                Step\# & Utente & Sistema \\
                \hline
                 2 a &  & Mostra schermata "Connessione assente" e termina caso d'uso\\
                 \hline 
        \end{tabularx} 
\end{table}
    
       


%Intestazione tabella%
\begin{table}[H]
    \caption{L'utente aggiunge una recensione}
    \begin{tabularx}{\textwidth}{|l|X|X|X|X|}
        \rowcolor{Gray}
      \hline Use Case \#5 & \multicolumn{4} {l|}{L'utente aggiunge una recensione} \\ \hline Goal in
      Context & \multicolumn{4}{>{\hsize=\dimexpr 4\hsize+4\tabcolsep+2\arrayrulewidth\relax}X|}{%
        L'utente vuole aggiungere una recensione} \\
     \hline Preconditions & \multicolumn{4}{>{\hsize=\dimexpr 4\hsize+4\tabcolsep+2\arrayrulewidth\relax}X|}{%
      } \\
     \hline Success End Conditions &
     \multicolumn{4}{>{\hsize=\dimexpr 4\hsize+4\tabcolsep+2\arrayrulewidth\relax}X|}{ L'utente aggiunge con successo una recensione ed il sistema tiene traccia di tale operazione} \\
     \hline Failed End Conditions &
     \multicolumn{4}{>{\hsize=\dimexpr 4\hsize+4\tabcolsep+2\arrayrulewidth\relax}X|}{Il server non è raggiungibile, l'utente non compila tutti i campi} \\
     \hline Primary Actor &
      \multicolumn{4}{l|}{Utente Autenticato} \\
     \hline Trigger & 
     \multicolumn{4}{>{\hsize=\dimexpr 4\hsize+4\tabcolsep+2\arrayrulewidth\relax}X|}{L'utente clicca sul \gls{Floating Action Button} e preme su "Aggiungi recensione" dalla schermata "Pagina Struttura"} \\
    \hline
    %Main Scenario%
    \multicolumn{5}{|>{\hsize=\dimexpr 4\hsize+4\tabcolsep+2\arrayrulewidth\relax}c|}{Main Scenario}\\\hline
    \end{tabularx}
    \setlength{\tabcolsep}{8pt}
    \renewcommand{\arraystretch}{1.5}
        \begin{tabularx}{\textwidth}{|c|X|X|}
            Step\# & Utente & Sistema \\
            \hline
             1 &  & Mostra schermata Aggiungi Recensione\\
             \hline
             2 &Compila correttamente tutti i campi della schermata& \\
             \hline          
             3 &Preme il tasto "Aggiungi"& \\
             \hline
             4 && Mostra la schermata "Aggiungi recensione success Dialog" e termina il caso d'uso\\
             \hline      
        \end{tabularx}
    \end{table}
    
    %Extension 1%
    \begin{table}[H]
    \caption{Aggiungi recensione- Estensione 1}
         \begin{tabularx}{\textwidth}{|c|X|X|}
                \hline
                \rowcolor{LightGray}
                \multicolumn{3}{|>{\hsize=\dimexpr 4\hsize+4\tabcolsep+2\arrayrulewidth\relax}c|}{Extension 1: il server non è raggiungibile}\\\hline
                Step\# & Utente & Sistema \\
                \hline
                 4 a &  & Mostra schermata "Aggiungi recensione errore dialog" e termina caso d'uso\\
                 \hline 
        \end{tabularx} 
    \end{table}
    %Extension 2%
    \begin{table}[H]
        \caption{Aggiungi recensione- Estensione 2}
             \begin{tabularx}{\textwidth}{|c|X|X|}
                    \hline
                    \rowcolor{LightGray}
                    \multicolumn{3}{|>{\hsize=\dimexpr 4\hsize+4\tabcolsep+2\arrayrulewidth\relax}c|}{Extension 1: l'utente non compila uno o più campi}\\\hline
                    Step\# & Utente & Sistema \\
                    \hline
                     1 b & Compila il form tralasciando uno o più campi & \\
                     \hline 
                     2 b & Preme il tasto "Aggiungi" & \\
                     \hline 
                     3 b &  & Mostra schermata "Aggiungi recensione empty dialog" e termina caso d'uso \\
                     \hline 
            \end{tabularx} 
        \end{table}
    
       

\pagebreak
%%%%% ===============================================================================
\section{Mockup}

The redaction of the thesis has to be carried on by the candidate indipendentely. A dissertation type thesis has the structure of a scientific article where it is required to derive, from the international literature, the most recent developments on the topic of interest, it is required to synthsize them, present them in an omogenous way, and finally compare the different approaches highlighting pros and cons of each of them. A sperimental type thesis has the structure of a scientific report, it faces a specific problem, typically within a more wide project of interest forthe supervisor, proposing a solution that is innovative if compared to the state of the art. A sperimental thesis also includes a validation of the proposed solution, made by means of experimental measuraments and/or numerical simulations.

%%%%% ===============================================================================
\section{Glossario}
\textbf{Navigation Drawer} = Menù a tendina laterale\\
\textbf{Floating Action Button (FAB)} = Bottone
\begin{figure}[H]
    \caption{In evidenza un Navigation Drawer ed un Floating Action Button}
    \includegraphics[scale=0.4]{Figures/DrawerUtenteL.png}
    \includegraphics[scale=0.4]{Figures/HomePageUtenteL.png} 
\end{figure}

%%%%% ===============================================================================

