% ===========================================================================
%
%		FEDERICO II THESIS TEMPLATE - ENGLISH
%  					* an example of Chapter 2: mathematical text
%	 
% 		AUTHOR:  		Antonio Esposito (antonio.esposito103@studenti.unina.it)
%		LAST UPDATED:	2017/06/20
%
% ===========================================================================

\chapter{Desgin}
\section{Descrizione del dominio}
In seguito ad un'analisi dei requisiti del sistema si è resa evidente l'esistenza di alcune entità
principali:
\begin{itemize}
    \item gli \textit{Utenti} che possono registrarsi, visualizzare delle \textit{Strutture}
    e scrivere delle recensioni
    \item degli \textit{Amministratori} i quali possono gestire i dati dei visitatori e le loro recensioni
\end{itemize}
%%%%% ===============================================================================
\section{Analisi dell'architettura}
Il sistema si basa principalmente su due pattern architetturali: Client-Server e Model-View-Controller.\\
Il sistema infatti è composto da diversi client che tramite internet interagiscono con un Server remoto.
\begin{center}
    \begin{figure}[H]
        \includegraphics[width=\textwidth]{Figures/Architettura client server.png}
        \caption{Architettura client server}
    \end{figure}
\end{center}
Per quanto riguarda l'architettura interna dei client si è deciso di utilizzare il modello MVC
favorendo la separazione dei componenti software del sistema ed in particolare separandone le responsabilità.
In questo modo si riesce ad ottenere alta coesione e basso accoppiamento, favorendo notevolemente
la manutenibilità.\\
In particolare:\\
la componente \textbf{Model} ha la responsabilità di fornire i metodi per accedere
ai dati utili all'applicazione, ha conoscenza del dominio applicativo ed è indipendente
dagli altri sottosistemi.\\
la componente \textbf{View} è responsabili della visualizzazione dei dati conenuti nel Model.\\
la componente \textbf{Controller} ha la responsabilità di gestire l'input dell'utente aggiornando di conseguenza
la View ed il Model.
\begin{center}
    \begin{figure}[H]
        \includegraphics[width=\textwidth]{Figures/MVC client.png}
        \caption{Architettura Model-View-Controller}
    \end{figure}
\end{center}
\section{Diagramma delle classi di design}

\section{CRC  Cards}

\section{Diagramma di stato di design}

\section{Diagramma di sequenza di design}