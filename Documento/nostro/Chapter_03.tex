% ===========================================================================
%
%		FEDERICO II THESIS TEMPLATE - ENGLISH
%  					* an example of Chapter 3: tables, figures and software code
%	 
% 		AUTHOR:  		Antonio Esposito (antonio.esposito103@studenti.unina.it)
%		LAST UPDATED:	2017/06/20
%
% ===========================================================================

\chapter{Testing}

The inclusion of tables and figures in a scientific publication follows
certain common and certain specific rules. Tables and figures are not
included inside the text but placed either on dedicated pages or floated
at the top or the bottom of a text page. \LaTeX\ handles floating
figures and tables automatically. Every table and figure must be
numbered and accompanied with a legend. The legend should describe the
contents of the table of figure with enough detail so that the reader
can understand them without studying the text of the publication. Each
table and figure should be refrenced by its number in the text. The
text should summarize the most important conclusions that can be drawn
from the table of figure. The text should be easy to follow and
understand even without seeingf the figures and tables (and on the
contrary, the figures and tables should be easy to understand even
without reading the text). Figures and tables should be referenced
indirectly in the sentences: instead of
\emph{``Table~\ref{tab03:Nejaka} shows that men are on average $9.9$
  kg heavier than women''} we write \emph{``Men are on average $9,9$
  kg heavier than women (see Table~\ref{tab03:Nejaka})}.


%%%%% ===============================================================================
\section{Test Plan per il System Testing}

\section{Codice jUnit per unit testing}