
%Intestazione tabella%
\begin{table}[H]
    \centering
    \footnotesize
    \caption{Test Case \#4}
    \begin{tabularx}{\textwidth}{|c|X|}
        \hline
        Applicativo & Mobile\\
        \hline
        Nome & Utente ricerca una struttura  \\
        \hline
        Descrizione & Test che assicura il corretto funzionamento della ricerca di una struttura\\
        \hline
        Note &  \\
        \hline
        Stato & Passato/Fallito\\
        \hline

    \end{tabularx}
    Continua alla pagina seguente
    \setlength{\tabcolsep}{8pt}
    \renewcommand{\arraystretch}{1.5}
\end{table}
%Step%
\begin{table}[H]
    \footnotesize
    \begin{tabularx}{\textwidth}{|c|X|X|X|}
        \hline
        Step\# & Input & Risultato Atteso & Risultato Ottenuto \\
        \hline
         1 & Viene aperta l'homepage dell'applicazione '
         & Viene visualizzato una schermata contenente una mappa con tutte le strutture esistenti in database
         &DA INSERIRE (Come ci si aspettava)\\
          \hline
        2 & Ci si sposta sulla mappa fino a visualizzare la struttura desiderata e ci si clicca sopra
        & Viene visualizzata la schermata della struttura
        & DA INSERIRE\\
         \hline 
        3 & Viene cliccato il tasto back in alto a sinistra
         & Viene mostrata l'homepage & DA INSERIRE\\
          \hline
        4 & Viene cliccato il tasto filtri in alto a destra
         & Viene visualizzata la schermata filtri
         & DA INSERIRE\\
          \hline 
        5 & Vengono inseriti i filtri: "Struttura", "Pompei", "Pub" e viene premuto il tasto "Cerca"
         & Viene visualizzata la lista delle strutture corrispondenti ai criteri di ricerca
         & DA INSERIRE\\
          \hline 
        6 & Viene cliccato il tasto back in alto a sinistra
          & Viene mostrata l'homepage
          & DA INSERIRE\\
           \hline      
        7 & Viene cliccato il tasto filtri in alto a destra
        & Viene visualizzata la schermata filtri
        & DA INSERIRE\\
        \hline       
        
        8 & Viene cliccato il tasto prossimità, viene inserito "100" nel textfield "Distanza massima" viene cliccato il tasto "Cerca"
        & Viene visualizzata la schermata contenete la lista delle strutture entro 100km di prossimità
        & DA INSERIRE\\
        \hline 
        
    \end{tabularx}
\end{table}
    
       
