
%Intestazione tabella%
\begin{table}[H]
    \centering
    \footnotesize
    \caption{Test Case \#2}
    \begin{tabularx}{\textwidth}{|c|X|}
        \hline
        Applicativo & Mobile\\
        \hline
        Nome & Utente accede alla piattaforma  \\
        \hline
        Descrizione & Test che assicura il corretto funzionamento della funzione di Login\\
        \hline
        Note &  L'utente non deve aver effettuato il login e deve esistere l'utente
        con nickname="nickname" e password "password"\\
        \hline
        Stato & Passato/Fallito\\
        \hline

    \end{tabularx}
    Continua alla pagina seguente
    \setlength{\tabcolsep}{8pt}
    \renewcommand{\arraystretch}{1.5}
\end{table}
%Step%
\begin{table}[H]
    \footnotesize
    \begin{tabularx}{\textwidth}{|c|X|X|X|}
        \hline
        Step\# & Input & Risultato Atteso & Risultato Ottenuto \\
        \hline
         1 & Dall'homepage viene cliccata l'iconda del menù in alto a sinistra 
         & Viene visualizzato il menù laterale sinsitra
         &DA INSERIRE (Come ci si aspettava)\\
          \hline
        2 & Viene cliccata la riga contenente "Login"
        & Viene visualizzata la schermata di Login con il bottone "Login" 2 TextField
        & DA INSERIRE\\
         \hline 
        3 & Non viene compilato nessun campo e viene premuto il tasto "Fine"
         & Viene mostrato un dialog di errore & DA INSERIRE\\
          \hline
        4 & Viene premuto il tasto "Ok"
         & Viene visualizzata nuovamente la schermata di Login
         & DA INSERIRE\\
          \hline 
          5 & Vengono inseriti i campi: "nickname" e "password" e viene premuto il tasto "Login"
         & Viene visualizzata mostrata la homepage dell'applicativo
         & DA INSERIRE\\
          \hline           
    \end{tabularx}
\end{table}
    
       
