
%Intestazione tabella%
\begin{table}[H]
    \centering
    \footnotesize
    \caption{Test Case \#4}
    \begin{tabularx}{\textwidth}{|c|X|}
        \hline
        Applicativo & Desktop\\
        \hline
        Nome & Amministratore elimina visitatore  \\
        \hline
        Descrizione & Test che assicura il corretto funzionamento dell'eliminazione di un visitatore'\\
        \hline
        Note & L'amministatore deve essere autenticato e deve esistere l'utente "utente"  \\
        \hline
        Stato & Passato/Fallito\\
        \hline

    \end{tabularx}
    Continua alla pagina seguente
    \setlength{\tabcolsep}{8pt}
    \renewcommand{\arraystretch}{1.5}
\end{table}
%Step%
\begin{table}[H]
    \footnotesize
    \begin{tabularx}{\textwidth}{|c|X|X|X|}
        \hline
        Step\# & Input & Risultato Atteso & Risultato Ottenuto \\
        \hline
         1 & Viene premuto il tasto "Visitatori" dalla schermata "Recensioni" 
         & Viene visualizzata la schermata "Visitatori" contententue una lista di visitatori
         &DA INSERIRE (Come ci si aspettava)\\
          \hline
        2 & Viene cliccata la riga contenente l'utente "utente"
        & Viene visualizzata la schermata di visualizzazione dell'utente contenente i bottoni "Elimina" e "Modifica"
          Sulla schermata sono presenti anche i seguenti textfield: nome, email, nickname, data di nascita, recensioni approvate e rifiutate .
        & DA INSERIRE\\
         \hline 
        3 & Viene cliccato il bottone "Elimina"
         & Viene visualizzato il dialog che ci informa che l'utente è stato eliminato con successo
         & DA INSERIRE\\
          \hline
          4 & Viene premuto il tasto "Ok"
         & Viene visualizzata visualizzata nuovamente la schermata con la lista dei visitatori
         & DA INSERIRE\\
          \hline      
    \end{tabularx}
\end{table}
    
       
