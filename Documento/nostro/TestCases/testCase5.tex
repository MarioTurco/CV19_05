
%Intestazione tabella%
\begin{table}[H]
    \centering
    \footnotesize
    \caption{Test Case \#5}
    \begin{tabularx}{\textwidth}{|c|X|}
        \hline
        Applicativo & Desktop\\
        \hline
        Nome & Amministratore modifica dati visitatore  \\
        \hline
        Descrizione & Test che assicura il corretto funzionamento della modifica dei dati di un visitatore'\\
        \hline
        Note & L'amministatore deve essere autenticato e deve esistere l'utente "utente" con nickname "user"  \\
        \hline
        Stato & Passato/Fallito\\
        \hline

    \end{tabularx}
    Continua alla pagina seguente
    \setlength{\tabcolsep}{8pt}
    \renewcommand{\arraystretch}{1.5}
\end{table}
%Step%
\begin{table}[H]
    \footnotesize
    \begin{tabularx}{\textwidth}{|c|X|X|X|}
        \hline
        Step\# & Input & Risultato Atteso & Risultato Ottenuto \\
        \hline
         1 & Viene premuto il tasto "Visitatori" dalla schermata "Recensioni" 
         & Viene visualizzata la schermata "Visitatori" contententue una lista di visitatori
         &Come ci si aspettava \\
          \hline
        2 & Viene cliccata la riga contenente l'utente "utente"
        & Viene visualizzata la schermata di visualizzazione dell'utente contenente i bottoni "Elimina" e "Modifica"
          Sulla schermata sono presenti anche i seguenti textfield: nome, email, nickname, data di nascita, recensioni approvate e rifiutate .
        & Come ci si aspettava\\
         \hline 
        3 & Viene cliccato il bottone "Modifica"
         & Viene visualizzata la schemata di modifica dei tasti utente contenente due editText ("Nickname" e "Password") e
         due bottoni ("Annulla" e "Conferma")
         & Come ci si aspettava\\
          \hline
        4 & Viene inserito il Nickname "user" e la nuova password: "pass"; viene premuto il tasto "Ok"
         & Viene visualizzata un dialog che avvisa che l'azione non ha avuto successo
         & Come ci si aspettava\\
          \hline  
          5 & Viene premuto il tasto "OK"
          & Viene visualizzata nuovamente la schermata contenente le informazioni di "utente"
          & Come ci si aspettava\\
          \hline      
        6 & Viene inserito il Nickname "nuovoNickname" e la nuova password: "pass"; viene premuto il tasto "Ok"
         & Viene visualizzata un dialog che avvisa che l'azione ha avuto successo
         & Come ci si aspettava\\
           \hline 
           7 & Viene premuto il tasto "Ok"
           & Viene visualizzata nuovamente la schermata con le informazioni dell'utente e
           viene visualizzato il nickname ("nuovoNickname")
           & Come ci si aspettava\\
             \hline                       
    \end{tabularx}
\end{table}
    
       
