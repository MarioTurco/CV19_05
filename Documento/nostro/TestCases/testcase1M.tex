
%Intestazione tabella%
\begin{table}[H]
    \centering
    \footnotesize
    \caption{Test Case \#1}
    \begin{tabularx}{\textwidth}{|c|X|}
        \hline
        Applicativo & Mobile\\
        \hline
        Nome & Utente si registra alla piattaforma  \\
        \hline
        Descrizione & Test che assicura il corretto funzionamento della funzione di registrazione\\
        \hline
        Note &  L'utente non deve aver effettuato il login e non deve esistere nessun utente
         registrato con i seguenti dati: nickname = "nick" email = "email.email@gmail.com".\\
        \hline
        Stato & Passato/Fallito\\
        \hline

    \end{tabularx}
    %Continua alla pagina seguente
    \setlength{\tabcolsep}{8pt}
    \renewcommand{\arraystretch}{1.5}
\end{table}
%Step%
\begin{table}[H]
    \footnotesize
    \begin{tabularx}{\textwidth}{|c|X|X|X|}
        \hline
        Step\# & Input & Risultato Atteso & Risultato Ottenuto \\
        \hline
         1 & Dall'homepage viene cliccata l'iconda del menù in alto a sinistra 
         & Viene visualizzato il menù laterale sinsitra
         &DA INSERIRE (Come ci si aspettava)\\
          \hline
        2 & Viene cliccata la riga contenente "Registrati"
        & Viene visualizzata la schermata di registrazione con il bottone "Fine", 5 TextField, 1 datePicker ed una checkbox
        & DA INSERIRE\\
         \hline 
        3 & Non viene compilato nessun campo e viene premuto il tasto "Fine"
         & Viene mostrato un dialog di errore & DA INSERIRE\\
          \hline
        4 & Viene premuto il tasto "Ok"
         & Viene visualizzata nuovamente la schermata di registrazione
         & DA INSERIRE\\
          \hline 
          5 & Vengono inseriti i campi: "email.email@gmail.com", "password","Mario",
          "Rossi", "17/11/1998", "nick" e viene premuto il tasto "Fine".
         & Viene visualizzata mostrato il dialog di registrazione effettuata
         & DA INSERIRE\\
          \hline           
    \end{tabularx}
\end{table}
    
       
