\subsection{Testing Black Box}
Effettueremo una serie di test, tramite strategia Strong Equivalence Class Testing sul seguente metodo:
\lstset{language=Java,
    keywordstyle=\color{magenta},
    basicstyle=\scriptsize\ttfamily,
    commentstyle=\ttfamily\itshape\color{gray},
    stringstyle=\ttfamily,
    showstringspaces=false,
    breaklines=true,
    frameround=ffff,
    frame=single,
    rulecolor=\color{black},
    numbers = left,
    extendedchars=true%,                      %questo rigo ed il successivo servono per usare le lettere à è ù nel codice
    %literate={à}{{\'a}}1 {è}{{\'a}}1 {ù}{{\'u}}1 
}
\begin{lstlisting}
public boolean tryLogin(String username, String password)
\end{lstlisting}
Definiamo le classi di equivalenza dei valori assumibili dai parametri in input:
%Intestazione tabella%
\begin{table}[H]
    \centering
    \footnotesize
    \caption{Classi di equivalenza di 'username'}
    \begin{tabularx}{\textwidth}{|c|X|}
        \hline
        Classe di Equivalenza & Descrizione\\
        \hline
        EC1 & username è null  \\
        \hline
        EC2 & username è una stringa vuota\\
        \hline
        EC3 &  username è sbagliato\\
        \hline
        EC4 & username è corretto\\
        \hline
    \end{tabularx}
    %Continua alla pagina seguente
    \setlength{\tabcolsep}{8pt}
    \renewcommand{\arraystretch}{1.5}
\end{table}
\begin{table}[H]
    \centering
    \footnotesize
    \caption{Classi di equivalenza di 'password'}
    \begin{tabularx}{\textwidth}{|c|X|}
        \hline
        Classe di Equivalenza & Descrizione\\
        \hline
        EC5 & password è null  \\
        \hline
        EC6 & password è una stringa vuota\\
        \hline
        EC7 &  password è sbagliata\\
        \hline
        EC8 & password è corretta\\
        \hline
    \end{tabularx}
    %Continua alla pagina seguente
    \setlength{\tabcolsep}{8pt}
    \renewcommand{\arraystretch}{1.5}
\end{table}
Effettuiamo il prodotto cartesiano tra le classi di equivalenza del primo parametro 
e quelle del secondo parametro per ottenere tutti i possibili testcase:
\begin{table}[H]
    \centering
    \footnotesize
    \caption{Lista dei TestCase}
    \begin{tabularx}{\textwidth}{|c|X|X|}
        \hline
        Test Case & Classe di Equivalenza & Descrizione\\
        \hline
        TC1 & EC1 & EC5  \\
        \hline
        TC2 & EC1 & EC6  \\
        \hline
        TC3 & EC1 & EC7  \\
        \hline
        TC4 & EC1 & EC8  \\
        \hline
        TC5 & EC2 & EC5  \\
        \hline
        TC6 & EC2 & EC6  \\
        \hline
        TC7 & EC2 & EC7  \\
        \hline
        TC8 & EC2 & EC8  \\
        \hline
        TC9 & EC3 & EC5  \\
        \hline
        TC10 & EC3 & EC6  \\
        \hline
        TC11 & EC3 & EC7  \\
        \hline
        TC12 & EC3 & EC8  \\
        \hline
        TC13 & EC4 & EC5  \\
        \hline
        TC14 & EC4 & EC6  \\
        \hline
        TC15 & EC4 & EC7  \\
        \hline
        TC16 & EC4 & EC8  \\
        \hline
    \end{tabularx}
    %Continua alla pagina seguente
    \setlength{\tabcolsep}{8pt}
    \renewcommand{\arraystretch}{1.5}
\end{table}

Si riporta di seguito il codice sorgente dei test:
\lstinputlisting{TestCases/AdminDAO_TestSuite.java}

       
