\pagebreak
%Intestazione tabella%
\begin{table}[H]
    \centering
    \footnotesize
    \caption{Test Case \#3}
    \begin{tabularx}{\textwidth}{|c|X|}
        \hline
        Applicativo & Desktop\\
        \hline
        Nome & Amministratore valuta una recensione  \\
        \hline
        Descrizione & Test che assicura il corretto funzionamento della valutazione delle recensioni\\
        \hline
        Note & L'amministatore deve essere autenticato e deve esistere una recensione scritta da "utente" per la struttura "Struttura" in data "GEN 1 2020" \\
        \hline
        Stato & Passato/Fallito\\
        \hline

    \end{tabularx}
    Continua alla pagina seguente
    \setlength{\tabcolsep}{8pt}
    \renewcommand{\arraystretch}{1.5}
\end{table}

\begin{table}[H]
    \footnotesize
    \begin{tabularx}{\textwidth}{|c|X|X|X|}
        \hline
        Step\# & Input & Risultato Atteso & Risultato Ottenuto \\
        \hline
         1 & Viene premuto il tasto "Recensioni" dalla schermata "Visitatori" 
         & Viene visualizzata la schermata "Recensioni" contententue una lista di recensioni
         &DA INSERIRE (Come ci si aspettava)\\
          \hline
        2 & Viene cliccata la riga contenente la recensione scritta in data "GEN 1 2020" dall'utente "utente" sulla struttura "Struttura"
        & Viene visualizzata la schermata della recensione contenete i tasti "Acceta" e  "Rifiuta", i textFiel contenenti il titolo, il testo e l'autore della descrizione.
        Viene visualizzato, in oltre, il numero di stelline della recensione.
        & DA INSERIRE\\
         \hline 
        3 & Viene cliccato il bottone "Accetta" o "Rifiuta"
         & Viene visualizzato il dialog che ci informa che l'azione è stata svolta con successo
         & DA INSERIRE\\
          \hline
          4 & Viene premuto il tasto "Ok"
         & Viene visualizzata la schermata Recensioni e nella lista delle recensioni non compare più 
         la recensione scritta dall'utente "Utente" sulla struttura "Struttura" in data "GEN 1 2020"
         & DA INSERIRE\\
          \hline      
    \end{tabularx}
\end{table}
    
       
