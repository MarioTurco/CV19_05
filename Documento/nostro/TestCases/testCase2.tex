
%Intestazione tabella%
\begin{table}[H]
    \footnotesize
    \caption{Test Case \#2}
    \begin{tabularx}{\textwidth}{|c|X|}
        \hline
        Applicativo & Desktop\\
        \hline
        Nome & Amministratore visualizza dati di un utente  \\
        \hline
        Descrizione & Test che assicura il corretto funzionamento della visualizzazione dei dati dell'utente\\
        \hline
        Note & L'amministratore deve essere loggato, deve esistere l'utente con nickname "utente" \\
        \hline
        Stato & Passato/Fallito\\
        \hline

    \end{tabularx}
    \setlength{\tabcolsep}{8pt}
    \renewcommand{\arraystretch}{1.5}
\end{table}

\begin{table}[H]
    \footnotesize
    \begin{tabularx}{\textwidth}{|c|X|X|X|}
        \hline
        Step\# & Input & Risultato Atteso & Risultato Ottenuto \\
        \hline
         1 & Viene premuto il tasto "Visitatori" dalla schermata "Recensioni" 
         & Viene visualizzata la schermata "Visitatori" contententue una lista di visitatori
         &Come ci si aspettava \\
          \hline
        2 & Viene cliccata la riga contenente l'utente con nickname "utente"
        & Viene visualizzata la schermata utente contenete i tasti "Modifica", "Elimina" ed i textfild nome, nickname, data di nascita, email, recensioni rifiutate ed approvate contenenti i relativi dati.
        & Come ci si aspettava\\
         \hline  
    \end{tabularx}
\end{table}
    
       
