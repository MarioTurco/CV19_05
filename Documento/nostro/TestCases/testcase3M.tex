
%Intestazione tabella%
\begin{table}[H]
    \centering
    \footnotesize
    \caption{Test Case \#3}
    \begin{tabularx}{\textwidth}{|c|X|}
        \hline
        Applicativo & Mobile\\
        \hline
        Nome & Utente aggiunge una recensione  \\
        \hline
        Descrizione & Test che assicura il corretto funzionamento dell'aggiunta di recensioni\\
        \hline
        Note &  L'utente deve aver effettuato il login\\
        \hline
        Stato & Passato/Fallito\\
        \hline

    \end{tabularx}
    Continua alla pagina seguente
    \setlength{\tabcolsep}{8pt}
    \renewcommand{\arraystretch}{1.5}
\end{table}
%Step%
\begin{table}[H]
    \footnotesize
    \begin{tabularx}{\textwidth}{|c|X|X|X|}
        \hline
        Step\# & Input & Risultato Atteso & Risultato Ottenuto \\
        \hline
         1 & Viene aperta la schermata di visualizzazione di una struttura
         & Viene visualizzato una schermata contenente tutti i dettagli della struttura
         &DA INSERIRE (Come ci si aspettava)\\
          \hline
        2 & Viene cliccato il pulsante "Aggiungi Recensione"
        & Viene visualizzata la schermata di aggiunta recensione
        & DA INSERIRE\\
         \hline 
        3 & Non viene compilato nessun campo e viene premuto il tasto "Fine"
         & Viene mostrato un dialog di errore & DA INSERIRE\\
          \hline
        4 & Viene premuto il tasto "Ok"
         & Viene visualizzata nuovamente la schermata di Login
         & DA INSERIRE\\
          \hline 
          5 & Vengono inseriti i campi: "titolo", "testo", viene cliccata la prima stellina a sinistra e viene premuto il tasto "Aggiungi"
         & Viene visualizzato dialog che notifica l'avvenuta operazione.
         & DA INSERIRE\\
          \hline 
          6 & Preme "Ok"
          & Viene visualizzato nuovamente la schermata della struttura
          & DA INSERIRE\\
           \hline                     
    \end{tabularx}
\end{table}
    
       
