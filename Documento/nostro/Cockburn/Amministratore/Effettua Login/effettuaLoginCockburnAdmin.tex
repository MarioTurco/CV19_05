
\begin{table}[h!]    
\def\arraystretch{1.5}
\caption{L'amministratore effettua il login }

%Intestazione tabella%
\begin{tabularx}{\textwidth}{|l|X|X|X|X|}
 \rowcolor{Gray}
  \hline Use Case \#1 & \multicolumn{4} {l|}{Effettua Login} \\ \hline Goal in
  Context & \multicolumn{4}{>{\hsize=\dimexpr 4\hsize+4\tabcolsep+2\arrayrulewidth\relax}X|}{%
    L'amministratore vuole accedere all'applicativo di Back-Office} \\
 \hline Preconditions & \multicolumn{4}{>{\hsize=\dimexpr 4\hsize+4\tabcolsep+2\arrayrulewidth\relax}X|}{%
 L'amministratore non ha effettuto lo use case "Effettua Login".  } \\
 \hline Success End Conditions &
 \multicolumn{4}{>{\hsize=\dimexpr 4\hsize+4\tabcolsep+2\arrayrulewidth\relax}X|}{ Il login dell'amministratore va a buon fine.} \\
 \hline Failed End Conditions &
 \multicolumn{4}{>{\hsize=\dimexpr 4\hsize+4\tabcolsep+2\arrayrulewidth\relax}X|}{I dati di login sono errati o il server non è raggiungibile.} \\
 \hline Primary Actor &
  \multicolumn{4}{l|}{Amministratore} \\
 \hline Trigger & 
 \multicolumn{4}{>{\hsize=\dimexpr 4\hsize+4\tabcolsep+2\arrayrulewidth\relax}X|}{L'amministratore preme il pulsante "Login" nella schermata LoginForm visibile all'avvio del software.} \\
\hline
%Main Scenario%
\multicolumn{5}{|>{\hsize=\dimexpr 4\hsize+4\tabcolsep+2\arrayrulewidth\relax}c|}{Main Scenario}\\\hline
\end{tabularx}
\setlength{\tabcolsep}{8pt}
\renewcommand{\arraystretch}{1.5}
    \begin{tabularx}{\textwidth}{|c|X|X|}
        Step\# & Amministratore & Sistema \\
        \hline
         1 &Compila correttamente i textField Username e Password  & \\
         \hline
         2 &Preme il pulsante "Login" dalla schermata LoginForm  & \\
         \hline
         3 &  &Mostra schermata HomePage\\
        \hline
    \end{tabularx} 
    \end{table}
    \begin{table}[H]
        \caption{Effettua Login - Estensione 1}
    %Extension 1%
    \begin{tabularx}{\textwidth}{|c|X|X|}
            \hline
            \rowcolor{LightGray}
            \multicolumn{3}{|>{\hsize=\dimexpr 4\hsize+4\tabcolsep+2\arrayrulewidth\relax}c|}{Extension 1: l'amministatore inserisce dati errati}\\\hline
            Step\# & Amministratore & Sistema \\
            \hline
             1 a &  Non complia o compila erroneamente i textField Username e Password& \\
             \hline
             2 a & Preme il pulsante Login & \\
             \hline
             3 a & & Mostra CredenzialiErrateDialog e termina caso d'uso \\
             \hline      
        \end{tabularx} 
    \end{table}
    \begin{table}[h!]
        \caption{Effettua Login - Estensione 2}
    %Extension 2%
    \begin{tabularx}{\textwidth}{|c|X|X|}
        \hline
        \rowcolor{LightGray}
        \multicolumn{3}{|>{\hsize=\dimexpr 4\hsize+4\tabcolsep+2\arrayrulewidth\relax}c|}{Extension 2: il server non è raggiungibile}\\\hline
        Step\# & Amministratore & Sistema \\
        \hline
         1 b &  Compila i campi username e password& \\
         \hline
         2 b & Preme il pulsante Login & \\
         \hline
         3 a & & Mostra 'Connessione assente dialog' e termina caso d'uso \\
         \hline
       
    \end{tabularx} 
\end{table}
