
\begin{table}[H]    
\def\arraystretch{1.5}
\caption{L'amministratore valuta una recensione}

%Intestazione tabella%
\begin{tabularx}{\textwidth}{|l|X|X|X|X|}
  \hline Use Case \#2 & \multicolumn{4} {l|}{Valuta Recensione} \\ \hline Goal in
  Context & \multicolumn{4}{>{\hsize=\dimexpr 4\hsize+4\tabcolsep+2\arrayrulewidth\relax}X|}{%
    L'amministratore valuta una recensione.} \\
 \hline Preconditions & \multicolumn{4}{>{\hsize=\dimexpr 4\hsize+4\tabcolsep+2\arrayrulewidth\relax}X|}{%
 L'amministratore ha effettuto lo use case "Effettua Login".  } \\
 \hline Success End Conditions &
 \multicolumn{4}{>{\hsize=\dimexpr 4\hsize+4\tabcolsep+2\arrayrulewidth\relax}X|}{ L'amministratore valuta una recensione. Il sistema tiene traccia di tale operazione.} \\
 \hline Failed End Conditions &
 \multicolumn{4}{>{\hsize=\dimexpr 4\hsize+4\tabcolsep+2\arrayrulewidth\relax}X|}{L'amministatore preme annulla. L'amministrazione valuta una recensione che è già stata valutata.} \\
 \hline Primary Actor &
  \multicolumn{4}{l|}{Amministratore} \\
 \hline Trigger & 
 \multicolumn{4}{>{\hsize=\dimexpr 4\hsize+4\tabcolsep+2\arrayrulewidth\relax}X|}{L'amministratore preme il pulsante "Recensioni" nella Homepage.} \\
\hline
%Main Scenario%
\multicolumn{5}{|>{\hsize=\dimexpr 4\hsize+4\tabcolsep+2\arrayrulewidth\relax}c|}{Main Scenario}\\\hline
\end{tabularx}
\setlength{\tabcolsep}{8pt}
\renewcommand{\arraystretch}{1.5}
    \begin{tabularx}{\textwidth}{|c|X|X|}
        Step\# & Amministratore & Sistema \\
        \hline
         1 &Preme il pulsante "Recensioni" nella schermata principale & \\
         \hline
         2 & & Mostra GestioneRecensioni\\
         \hline
         3 & Clicca sul radio button accanto ad una recensione e preme il pulsante "Conferma" &\\
         \hline
         4 & & Mostra ValutaRecensione\\
       \hline
         5 & Clicca sul pulsante Approva &\\
        \hline
        6& &Mostra RecensioneApprovataDialog e termina lo use case\\
        \hline
    \end{tabularx}
    %Extension 1%
\end{table}
\begin{table}[h!]
    \caption{Valuta una recensione - Estensione 1}
        \begin{tabularx}{\textwidth}{|c|X|X|}
            \hline
            \multicolumn{3}{|>{\hsize=\dimexpr 4\hsize+4\tabcolsep+2\arrayrulewidth\relax}c|}{Extension 1: l'amministatore rifiuta una recensione}\\\hline
            Step\# & Amministratore & Sistema \\
            \hline
             5 a &Preme il pulsante "Rifiuta". & \\
             \hline
             6 a & & Mostra RecensioneEliminataDialog.\\
            \hline
        \end{tabularx}
\setlength{\tabcolsep}{8pt}
\renewcommand{\arraystretch}{1.5}
\end{table}

 %Extension 2%
\begin{table}[h!]
    \caption{Valuta una recensione - Estensione 2}
    \begin{tabularx}{\textwidth}{|c|X|X|}
        \hline
        \multicolumn{3}{|>{\hsize=\dimexpr 4\hsize+4\tabcolsep+2\arrayrulewidth\relax}c|}{Extension 2: l'amministatore preme annulla}\\\hline
        Step\# & Amministratore & Sistema \\
        \hline
         3/5 b &Preme il pulsante "Annulla". & \\
         \hline
         4/6 b & & Ritorna alla schermata principale e termina il caso d'uso.\\
        \hline
    \end{tabularx}
\end{table}
%Estensione 3
\begin{table}[h!]
    \caption{Valuta una recensione - Estensione 3}
     \begin{tabularx}{\textwidth}{|c|X|X|}
        \hline
        \multicolumn{3}{|>{\hsize=\dimexpr 4\hsize+4\tabcolsep+2\arrayrulewidth\relax}c|}{Extension 3: la recensione è già stata valutata }\\\hline
         Step\# & Amministratore & Sistema \\
         \hline
          6 c  & & Mostra Fallimento Dialog \\
          \hline
          7 c & Preme Ok & \\
          \hline
          8 c &  & Ritorna alla schermata principale e termina il caso d'uso.\\
         \hline
     \end{tabularx}
\end{table}
\paragraph{Nota 1} Dato che il sistema può essere gestito da più amministratori è possibile che due di questi aprano contemporaneamente
la schermata di valutazione della stessa Recensione. In questo caso andrà a buon fine soltato la valutazione dell'amministatore
che per primo la rifiuterà o confermerà. L'altro amministratore vedrà la schermata di errore come descritto nella estensione 3. 
