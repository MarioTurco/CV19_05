\pagebreak
\begin{table}[H]    
    \def\arraystretch{1.5}
    \caption{L'amministratore modifica i dati di un visitatore}
    
    %Intestazione tabella%
    \begin{tabularx}{\textwidth}{|l|X|X|X|X|}
        \rowcolor{Gray}
      \hline Use Case \#4 & \multicolumn{4} {l|}{Modifica i dati visitatore} \\ \hline Goal in
      Context & \multicolumn{4}{>{\hsize=\dimexpr 4\hsize+4\tabcolsep+2\arrayrulewidth\relax}X|}{%
        L'amministratore vuole modificare lo username e la password di un visitatore} \\
     \hline Preconditions & \multicolumn{4}{>{\hsize=\dimexpr 4\hsize+4\tabcolsep+2\arrayrulewidth\relax}X|}{%
     L'amministratore ha effettuto lo use case "Effettua Login".  } \\
     \hline Success End Conditions &
     \multicolumn{4}{>{\hsize=\dimexpr 4\hsize+4\tabcolsep+2\arrayrulewidth\relax}X|}{ L'amministratore modifica con successo i dati di un visitatore} \\
     \hline Failed End Conditions &
     \multicolumn{4}{>{\hsize=\dimexpr 4\hsize+4\tabcolsep+2\arrayrulewidth\relax}X|}{Il server non è raggiungibile o l'amministratore non compila correttamente tutti i campi} \\
     \hline Primary Actor &
      \multicolumn{4}{l|}{Amministratore} \\
     \hline Trigger & 
     \multicolumn{4}{>{\hsize=\dimexpr 4\hsize+4\tabcolsep+2\arrayrulewidth\relax}X|}{L'amministratore preme il pulsante "Modifica" dalla Schermata "Visualizza Visitatore" .} \\
    \hline
    %Main Scenario%
    \multicolumn{5}{|>{\hsize=\dimexpr 4\hsize+4\tabcolsep+2\arrayrulewidth\relax}c|}{Main Scenario}\\\hline
    \end{tabularx}
\end{table}
\begin{table}[h!]
    \setlength{\tabcolsep}{8pt}
    \renewcommand{\arraystretch}{1.5}
        \begin{tabularx}{\textwidth}{|c|X|X|}
            \rowcolor{Gray}
            \hline
            Step\# & Amministratore & Sistema \\
            \hline
             1 &Preme sul tasto "Modifica" dalla schermata "Visualizza Visitatore" & \\
             \hline
             2 & & Mostra schermata "Modifica Visitatore" \\
             \hline
             3 & Compila correttamente tutti i campi di testo&  \\
             \hline
             4 & Preme il tasto "Conferma" & \\
             \hline
             5 & & Mostra schermata "Modifica Success Dialog" e termina caso d'uso \\
             \hline
        \end{tabularx}
    \end{table}
    %Extension 1%
    \begin{table}[h!]
        \caption{Modifica dati visitatore - Estensione 1}
            \begin{tabularx}{\textwidth}{|c|X|X|}
                \hline
                \rowcolor{LightGray}
                \multicolumn{3}{|>{\hsize=\dimexpr 4\hsize+4\tabcolsep+2\arrayrulewidth\relax}c|}{Extension 1: il server non è raggiungibile}\\\hline
                Step\# & Amministratore & Sistema \\
                \hline
                1a &Preme sul tasto "Modifica" dalla schermata "Visualizza Visitatore" & \\
                \hline
                2a & & Mostra schermata "Connessione Assente Dialog" \\
             \hline
            \end{tabularx}
    \end{table}
    %Extension 2%
    \begin{table}[h!]
        \caption{Modifica dati visitatore - Estensione 2}
            \begin{tabularx}{\textwidth}{|c|X|X|}
                \hline
                \rowcolor{LightGray}
                \multicolumn{3}{|>{\hsize=\dimexpr 4\hsize+4\tabcolsep+2\arrayrulewidth\relax}c|}{Extension 2: l'amministratore compila erroneamente i campi di testo}\\\hline
                Step\# & Amministratore & Sistema \\
                \hline
                3b &Non compila, compila parzialmente o compila in modo errato i campi di testo & \\
                \hline
                2a & & Mostra schermata "Password o username invalidi dialog" \\
             \hline
            \end{tabularx}
    \end{table}
    
    