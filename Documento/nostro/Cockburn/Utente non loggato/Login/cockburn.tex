
%Intestazione tabella%
\begin{table}[H]
\caption{Effettua Login - Main Scenario}
\begin{tabularx}{\textwidth}{|l|X|X|X|X|}
  \rowcolor{Gray}
  \hline Use Case \#1 & \multicolumn{4} {l|}{Utente effettua Login} \\ \hline Goal in
  Context & \multicolumn{4}{>{\hsize=\dimexpr 4\hsize+4\tabcolsep+2\arrayrulewidth\relax}X|}{%
    L'utente vuole effettuare l'accesso al suo account} \\
 \hline Preconditions & \multicolumn{4}{>{\hsize=\dimexpr 4\hsize+4\tabcolsep+2\arrayrulewidth\relax}X|}{%
 L'utente non ha effettuato lo use case Effettua Login  } \\
 \hline Success End Conditions &
 \multicolumn{4}{>{\hsize=\dimexpr 4\hsize+4\tabcolsep+2\arrayrulewidth\relax}X|}{ Il login dell'utente va a buon fine.} \\
 \hline Failed End Conditions &
 \multicolumn{4}{>{\hsize=\dimexpr 4\hsize+4\tabcolsep+2\arrayrulewidth\relax}X|}{I dati di login sono errati oppure il server non è raggiungibile.} \\
 \hline Primary Actor &
  \multicolumn{4}{l|}{Utente non loggato} \\
 \hline Trigger & 
 \multicolumn{4}{>{\hsize=\dimexpr 4\hsize+4\tabcolsep+2\arrayrulewidth\relax}X|}{L'utente preme il pulsante 'Login' nel Navigation Drawer laterale dalla schermata 'HomePage utente non loggato'.} \\
\hline
%Main Scenario%
\multicolumn{5}{|>{\hsize=\dimexpr 4\hsize+4\tabcolsep+2\arrayrulewidth\relax}c|}{Main Scenario}\\\hline
\end{tabularx}
\setlength{\tabcolsep}{8pt}
\renewcommand{\arraystretch}{1.5}
    \begin{tabularx}{\textwidth}{|c|X|X|}
        Step\# & Utente & Sistema \\
        \hline
         1 &Compila correttamente i textField Username e Password  & \\
         \hline
         2 &Preme il pulsante "Login" dalla schermata LoginForm  & \\
         \hline
         3 &  &Mostra schermata HomePage\\
        \hline
    \end{tabularx}
  \end{table}
  %Extension 1%
  \begin{table}[h!]
    \caption{Effettua Login - Estensione 1}
        \begin{tabularx}{\textwidth}{|c|X|X|}
            \hline
            \rowcolor{LightGray}
            \multicolumn{3}{|>{\hsize=\dimexpr 4\hsize+4\tabcolsep+2\arrayrulewidth\relax}c|}{Extension 1: l'utente inserisce dati errati}\\\hline
            Step\# & Utente & Sistema \\
            \hline
             1 a &  Compila erroneamente i textField Username e Password& \\
             \hline
             2 a & Preme il pulsante Login & \\
             \hline
             3 a & & Mostra schermata Login Dati Errati Dialog e termina caso d'uso \\
             \hline        
        \end{tabularx}
      \end{table}
      %Extension 2%
      \begin{table} [h!]
      \caption{Effettua Login - Estensione 2}
    \begin{tabularx}{\textwidth}{|c|X|X|}
      \hline
      \rowcolor{LightGray}
      \multicolumn{3}{|>{\hsize=\dimexpr 4\hsize+4\tabcolsep+2\arrayrulewidth\relax}c|}{Extension 2: l'utente non compila tutti i campi}\\\hline
      Step\# & Utente & Sistema \\
      \hline
       1 b &  Non compila o compila solo un tra i textField Username e Password& \\
       \hline
       2 b & Preme il pulsante Login & \\
       \hline
       3 b & & Mostra schermata Login campi vuoti Dialog e termina caso d'uso\\
       \hline        
  \end{tabularx}
\end{table}
\pagebreak
%Extension 3%
\begin{table}[h!]
  \caption{Effettua Login - Estensione 3}
  \begin{tabularx}{\textwidth}{|c|X|X|}
  \hline
  \rowcolor{LightGray}
  \multicolumn{3}{|>{\hsize=\dimexpr 4\hsize+4\tabcolsep+2\arrayrulewidth\relax}c|}{Extension 3: il server risulta non raggiungibile}\\\hline
  Step\# & Utente & Sistema \\
  \hline
   1 c & Compila correttamente tutti i campi& \\
   \hline
   2 c & Preme il tasto "Login" & \\
  \hline
   3 c & & Mostra schermata "Login server irraggiungibile dialog" e termina caso d'uso\\
   \hline
\end{tabularx} 
\end{table}
