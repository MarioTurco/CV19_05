
%Intestazione tabella%
\setcounter{table}{0}
\begin{table}[H]
    \footnotesize
    \caption{Test Case \#1}
    \begin{tabularx}{\textwidth}{|c|X|}
        \hline
        Applicativo & Desktop\\
        \hline
        Nome & Amministratore effettua login  \\
        \hline
        Descrizione & Testo che assicura il corretto funzionamento del login dell'amministratore\\
        \hline
        Note & Esiste un amministratore con le seguenti credenziali: "admin", "admin" \\
        \hline
        Stato & Passato/Fallito\\
        \hline

    \end{tabularx}
    \setlength{\tabcolsep}{8pt}
    \renewcommand{\arraystretch}{1.5}
\end{table}

\begin{table}[H]
    \footnotesize
    \begin{tabularx}{\textwidth}{|c|X|X|X|}
        \hline
        Step\# & Input & Risultato Atteso & Risultato Ottenuto \\
        \hline
         1 & Viene aperto l'applicativo Desktop  
         & Viene visualizzata la schermata "Login" contententue due Textbox (Username e Password) 
         ed un tasto
         &DA INSERIRE (Come ci si aspettava)\\
          \hline
        2 & Vengono inserite delle credenziali \textbf{errate} "0000", "0000" e viene cliccato 
        il tasto "Login" 
        & Il login fallisce e viene visualizzata la schermata "Credeziali errate Dialog" 
        & DA INSERIRE\\
         \hline
        3 & Vengono inserite le credenziali "admin", "admin" e viene cliccato 
        il tasto "Login" 
        & Il login ha successo e viene visualizzata la schermata "
        Revisioni" 
        & DA INSERIRE\\
         \hline
                 
    \end{tabularx}
\end{table}
    
       
