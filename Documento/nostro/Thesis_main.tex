% ===========================================================================
%
%		FEDERICO II THESIS TEMPLATE - ENGLISH
%  					* the main file
%	 
% 		AUTHOR:  		Antonio Esposito (antonio.esposito103@studenti.unina.it)
%		LAST UPDATED:	2017/06/17
%		ORIGINAL FILE:	LaTeX template of Charles University in Prague
%
% ===========================================================================

%		Single page layout:
\documentclass[12pt, a4paper]{report}
\let\openright=\clearpage

%		Double page layout
% \documentclass[12pt, a4paper, twoside, openright]{report}
% \let\openright=\cleardoublepage


%
%		Additional useful packages
%
\usepackage[T1]{fontenc}			% output font encoding
\usepackage[utf8]{inputenc}			% accented letters from keyboard
\usepackage[italian]{babel}		% document languages (the last is the main one)
\usepackage{setspace}				
\usepackage{multicol}				% multi-column layout
\usepackage{url}					% uniform resource locator
\usepackage{xcolor,colortbl}
\usepackage[hidelinks]{hyperref}
\usepackage{tabularx}       
\usepackage{amsfonts}             
\usepackage{pdfpages}     
\usepackage{float}  
\usepackage{hyperref}
\hypersetup{
    colorlinks=true,
    linkcolor=blue,
    filecolor=magenta,      
    urlcolor=cyan,
}  
\usepackage{fancyvrb}       
\usepackage{pdfpages}      
\usepackage{booktabs}         
\usepackage{listings}
\usepackage{glossaries}
\usepackage{indentfirst}  
\usepackage[nottoc]{tocbibind}
\newcommand{\FIGDIR}{./Figures}    		% directory containing figures
\addto\captionsitalian{% Replace "english" with the language you use
  \renewcommand{\contentsname}%
    {Indice}%
}
\definecolor{javared}{rgb}{1.0,0.33,0.64} % for strings
\definecolor{javagreen}{rgb}{0.0,0.5,0.0} % comments
\definecolor{javapurple}{rgb}{0.01,0.28, 1.0} % keywords
\definecolor{javadocblue}{rgb}{0.25,0.35,0.75} % javadoc
 
\lstset{language=Java,
basicstyle=\scriptsize\ttfamily,
keywordstyle=\color{javapurple}\bfseries,
stringstyle=\color{javared}\ttfamily,
commentstyle=\color{javagreen}\ttfamily\itshape,
morecomment=[s][\color{javadocblue}]{/**}{*/},
numbers=left,
breaklines=true,
frame=single,
numberstyle=\tiny\color{black},
tabsize=3,
showspaces=false,
showstringspaces=false}
%%COLORI%%
\definecolor{LightGray}{gray}{0.95}
\definecolor{Gray}{gray}{0.75}
\pdfobjcompresslevel=0
%%%%% -----------------------------------------------------------

\newcommand{\T}[1]{#1^\top}        

%\newcommand{\goto}{\rightarrow}
%\newcommand{\gotop}{\stackrel{P}{\longrightarrow}}
%\newcommand{\maon}[1]{o(n^{#1})}
%\newcommand{\abs}[1]{\left|{#1}\right|}
%\newcommand{\dint}{\int_0^\tau\!\!\int_0^\tau}
%\newcommand{\isqr}[1]{\frac{1}{\sqrt{#1}}}

\newcommand{\pulrad}[1]{\raisebox{1.5ex}[0pt]{#1}}
\newcommand{\mc}[1]{\multicolumn{1}{c}{#1}}

\makeglossaries
\newglossaryentry{Sale}
{
    name=Sale,
    description={In crittografia, un sale è una sequenza casuale di bit utilizzata assieme ad una password come input a una funzione unidirezionale, di solito una funzione hash, il cui output è conservato al posto della sola password, e può essere usato per autenticare gli utenti. }
}
\newglossaryentry{CRUD}{ 
    name=CRUD, 
    description={ Le operazioni CRUD (Create, Read, Update, Delete) sono le principali operazioni di gestione dei dati su una base di dati}
}
\newglossaryentry{Floating Action Button}{ 
    name=Floating Action Button, 
    description={ Bottone circolare spesso posto in basso a destra nella schermata di un applicazione}
}

%%%	MAIN DOCUMENT 
%
%

\begin{document}
 
% ===========================================================================
%
%		FEDERICO II THESIS TEMPLATE - ENGLISH
%  					* the header
%
%		AUTHOR:  		Antonio Esposito (antonio.esposito103@studenti.unina.it)
%		LAST UPDATED:	2017/06/20
%
% ===========================================================================

%
%%%		TITLE PAGE
%
\pagestyle{empty}
\begin{center}

% TO EDIT ACCORDINGLY TO...
% -----------------------------------------------------------------------------------------------
%

% ... University Name
{\bfseries\Huge Università degli Studi di Napoli\\}
\vspace{2.54mm}
{\bfseries\Huge Federico II\\}
\vspace{5mm}

% ... Logo
\centerline{\mbox{\includegraphics[width=36mm]{\FIGDIR/fiilogo}}}

% ... Departement
\medskip
{\bfseries\LARGE Dipartimento di Ingegneria Elettrica e\\}
\vspace{2.54mm}
{\bfseries\LARGE delle Tecnologie dell'Informazione\\}
\vspace{2.54mm}

% ... Degree Class
{\emph{\large Classe di Laurea di Scienze e tecnologie Informatiche,\\}}
{\emph{\large Classe n. L-31\\}}
\vspace{2.54mm}

% ... Degree Course
{\large Corso di Laurea Triennale in Informatica\\}
\vspace{5mm}

% Change the name of Departement, Class and Cource in English?

%
% -----------------------------------------------------------------------------------------------
%

\vfill
{\Large Progetto\\}
\vspace{4mm}
{\emph{\LARGE Cerca Viaggi 2019\\}}		% Thesis title
\vspace{4mm}

\vfill

\begin{multicols}{2}
	{\large Docente:\\}
	Prof. Di Martino Sergio\\
	\vspace{10mm}
	\columnbreak
	{\large Candidati:\\}
	Turco Mario\\
	Matr. N86002503\\
	Longobardi Francesco\\
	Matr. N86002468\\
	Rauso Giuseppe\\
	Matr. N86002481
	\vspace{10mm}
\end{multicols}

\vfill

% Fill the year
{\large Academic Year\\ 2019/2020}

\end{center}


\pagenumbering{roman}

%
%%%		TABLE OF CONTENTS AND OTHER
%
\newpage
\openright
{\hypersetup{hidelinks}
\tableofcontents
}


%
%%%		Optional
%

%\listoffigures
%\listoftables

%\chapter*{List of abbreviations}
%\addcontentsline{toc}{chapter}{List of abbreviations}
\pagenumbering{arabic}

% ===========================================================================
%
%		FEDERICO II THESIS TEMPLATE - ENGLISH
%  					* an example of Introduction
%	 
% 		AUTHOR:  		Antonio Esposito (antonio.esposito103@studenti.unina.it)
%		LAST UPDATED:	2017/06/20
%
% ===========================================================================

\chapter*{Introduction}
\addcontentsline{toc}{chapter}{Introduction}

The introduction is prior to any chapter. 
It has to present the problem taken into exam and clarify the aim of the work, eventually pointing out the original contributions to the scientific literature; the introduction can cointain a brief summary of the thesis organization. The following chapters include the description of the state of the art concerning the discussed topic. For experimental theses, some chapters are dedicated to the description of the innovative solution and its validation. The conclusions, finally, must sum up briefly the faced challenges and report the main consideration that can be done after the experimental results.

The manuscript must include a bibliography that lists all the references (books, scientific articles, theses, web sites, etc.), together with proper information about where the reader can find the cited material. See the references of this file to understand the desired format for the citations.

In the following part of this file, more details on the preparation and presentation of the thesis are given in Chapter 1. Then, examples of typesetting of mathematical text are shown in Chapter 2. Table, figue, and software code are instead discussed in Chapter 3. Conclusions follow. The aim of the present file is to show an example of thesis structure, and, in its \LaTeX version, the main commands that can be useful during a thesis writing are used: in this way, the student that is willing to write his/her thesis with \LaTeX, has a good starting point. It's worth noting that the current version of the file was written by an italian student, and this means that lots of errors are present, expecially regarding the english grammar\dots

%Revisioni
\part{Documento dei Requisiti Software}
% ===========================================================================
%
%		FEDERICO II THESIS TEMPLATE - ENGLISH
%  					* an example of Chapter 1: information about the discussion of the thesis
%	 
% 		AUTHOR:  		Antonio Esposito (antonio.esposito103@studenti.unina.it)
%		LAST UPDATED:	2017/06/20
%
% ===========================================================================
\chapter{Introduzione}
Lo scopo di questo documento è quello di specificare i requisiti del sistema software "Cerca Viaggi" per facilitarne la realizzazione e la validazione ed, in particolare,
si vogliono formalizzare i requisiti funzionali e non funzionali del sistema. \\Il documento prevede vari livelli di raffinamento partendo dal linguaggio naturale
ed arrivando ad un linguaggio strutturato ed a modelli UML.
\section{Requisiti Funzionali}
\begin{enumerate}
    \item  \textbf{Accesso amministratore al sistema}: un amministratore dovrà poter accedere al sistema tramite email e password (usando l'applicativo dekstop).
    \item  \textbf{Amministratore valuta una recensione}: un amministatore dovrà poter accettare o rifiutare una recensione scritta da un utente decidendo quindi se questa verrà pubblicata o eliminata.
    \item \textbf{Accesso utente al sistema}: un utente, dopo aver effettuato la registrazione, dovrà poter accedere al suo account tramite email e password (usando l'applicativo mobile). 
    \item  \textbf{Registrazione di un nuovo utente}: l'applicativo mobile dovrà permettere ad un utente, non precedentemente registrato, di creare un account specificando 
    i propri dati anagrafici (nome, cognome e data di nascita), i dati per effettuare l'accesso (email e passowrd) ed un nickname da utilizzare, secondo volonta dell'utente, per firmare  
    le proprie recensioni
    \item \textbf{Amministratore modifica dati dei visitatori}: un amministratore dovrà poter modificare i dati di un utente in particolare, email e password
    \item \textbf{Amministratore elimina account utente}: un amministratore dovrà poter eliminare l'account di un utente
    \item \textbf{Amministratore visualizza dati di un visitatore}: un amministratore dovrà poter visualizzare i dati anagrafici di un visitatore (nome, cognome e data di nascita), 
    il nickname ed il numero di recensioni approvate e rifiutate.
    \item \textbf{Utente visualizza struttra}: un utente dovrà poter visualizzare i dettagli di una struttura
    \item \textbf{Utente aggiunge una recensione}: un utente, una volta aver effettuato l'accesso, dovrà poter aggiungere una recensione ad una struttura
    \item \textbf{Utente visualizaza dettagli recensione}: un utente dovrà poter leggere i dettagli di una recensione, in particolare leggendone l'autore, la data di inserimento, la valutazione
    ed il testo integrale.
\end{enumerate}
\section{Requisiti non funzionali}
\begin{enumerate}
    \item Il sistema deve garantire una buona usabilità per tutti coloro che ne usufruiscono.
\end{enumerate}
\chapter{Modello funzionale}

This chapter contains useful information for the preparation and the presentation of the master degree thesis for students of Electronic Engineering (M61), at the University of Study of Naples Federico II.

The final test for the Master Degree course in Electronic Engineering consists in the preparation and discussion of a thesis, written with the help of a supervisor (eventually with one or two co-supervisors). This work is the final result of the student career and it testifies his/her ability in exploring in deep the topics encountered during the degree course.

%%%%% ===============================================================================
\section{Modellazione dei casi d'uso}

The supervisor is one of the professors that the candidate encountered during the degree course. Usually, the student finds its supervisor through informal talks, once provided that the professor is available and the student is interested in the professor's topics of interest. The degree course, on its website \url{www.ingegneria-elettronica.unina.it}, has defined a page with a non-esaustive list of available theses topics, in order to facilitate the information exchange between students and professors.

In case the thesis is developed after an intra-moenia internship, among one of the laboratories of the departement, the tutor that has already followed the student during the internship becomes the supervisor.

The supervisor defines the thesis topic. As already mentioned, the supervisor can be helped by one (or two maximum) co-supervisor. Supervisor and co-supervisor must guide and assist the student during the thesis developement and also provide him all the needed methodological and practical instruments. During the thesis are usually foreseen periodical meetings of the candidate with the supervisor, during which the ongoing work and the obtained results are discussed, also to define the future steps of the work.


%%%%% ===============================================================================
\section{Tabelle di Cockburn}
Di seguito si riportano, divise per attori, le tabelle di Cockburn relative agli Use Case Diagram.
\subsection{Amministratore}

\begin{table}[h!]    
\def\arraystretch{1.5}
\caption{L'amministratore effettua il login }

%Intestazione tabella%
\begin{tabularx}{\textwidth}{|l|X|X|X|X|}
 \rowcolor{Gray}
  \hline Use Case \#1 & \multicolumn{4} {l|}{Effettua Login} \\ \hline Goal in
  Context & \multicolumn{4}{>{\hsize=\dimexpr 4\hsize+4\tabcolsep+2\arrayrulewidth\relax}X|}{%
    L'amministratore vuole accedere all'applicativo di Back-Office} \\
 \hline Preconditions & \multicolumn{4}{>{\hsize=\dimexpr 4\hsize+4\tabcolsep+2\arrayrulewidth\relax}X|}{%
 L'amministratore non ha effettuto lo use case "Effettua Login".  } \\
 \hline Success End Conditions &
 \multicolumn{4}{>{\hsize=\dimexpr 4\hsize+4\tabcolsep+2\arrayrulewidth\relax}X|}{ Il login dell'amministratore va a buon fine.} \\
 \hline Failed End Conditions &
 \multicolumn{4}{>{\hsize=\dimexpr 4\hsize+4\tabcolsep+2\arrayrulewidth\relax}X|}{I dati di login sono errati o il server non è raggiungibile.} \\
 \hline Primary Actor &
  \multicolumn{4}{l|}{Amministratore} \\
 \hline Trigger & 
 \multicolumn{4}{>{\hsize=\dimexpr 4\hsize+4\tabcolsep+2\arrayrulewidth\relax}X|}{L'amministratore preme il pulsante "Login" nella schermata LoginForm visibile all'avvio del software.} \\
\hline
%Main Scenario%
\multicolumn{5}{|>{\hsize=\dimexpr 4\hsize+4\tabcolsep+2\arrayrulewidth\relax}c|}{Main Scenario}\\\hline
\end{tabularx}
\setlength{\tabcolsep}{8pt}
\renewcommand{\arraystretch}{1.5}
    \begin{tabularx}{\textwidth}{|c|X|X|}
        Step\# & Amministratore & Sistema \\
        \hline
         1 &Compila correttamente i textField Username e Password  & \\
         \hline
         2 &Preme il pulsante "Login" dalla schermata LoginForm  & \\
         \hline
         3 &  &Mostra schermata HomePage\\
        \hline
    \end{tabularx} 
    \end{table}
    \begin{table}[H]
        \caption{Effettua Login - Estensione 1}
    %Extension 1%
    \begin{tabularx}{\textwidth}{|c|X|X|}
            \hline
            \rowcolor{LightGray}
            \multicolumn{3}{|>{\hsize=\dimexpr 4\hsize+4\tabcolsep+2\arrayrulewidth\relax}c|}{Extension 1: l'amministatore inserisce dati errati}\\\hline
            Step\# & Amministratore & Sistema \\
            \hline
             1 a &  Non complia o compila erroneamente i textField Username e Password& \\
             \hline
             2 a & Preme il pulsante Login & \\
             \hline
             3 a & & Mostra CredenzialiErrateDialog e termina caso d'uso \\
             \hline      
        \end{tabularx} 
    \end{table}
    \begin{table}[h!]
        \caption{Effettua Login - Estensione 2}
    %Extension 2%
    \begin{tabularx}{\textwidth}{|c|X|X|}
        \hline
        \rowcolor{LightGray}
        \multicolumn{3}{|>{\hsize=\dimexpr 4\hsize+4\tabcolsep+2\arrayrulewidth\relax}c|}{Extension 2: il server non è raggiungibile}\\\hline
        Step\# & Amministratore & Sistema \\
        \hline
         1 b &  Compila i campi username e password& \\
         \hline
         2 b & Preme il pulsante Login & \\
         \hline
         3 a & & Mostra 'Connessione assente dialog' e termina caso d'uso \\
         \hline
       
    \end{tabularx} 
\end{table}

\documentclass[a4paper]{article}
\usepackage[T1]{fontenc}
\usepackage[utf8]{inputenc}
\usepackage{geometry}
\usepackage{multirow}
\usepackage{float} 
\usepackage{tabularx}


\geometry{
  left=30mm,
  right=30mm
}

\begin{document}
\begin{table}[H]    
\def\arraystretch{1.5}


%Intestazione tabella%
\begin{tabularx}{\textwidth}{|l|X|X|X|X|}
  \hline Use Case \#2 & \multicolumn{4} {l|}{Valuta Recensione} \\ \hline Goal in
  Context & \multicolumn{4}{>{\hsize=\dimexpr 4\hsize+4\tabcolsep+2\arrayrulewidth\relax}X|}{%
    L'amministratore valuta una recensione.} \\
 \hline Preconditions & \multicolumn{4}{>{\hsize=\dimexpr 4\hsize+4\tabcolsep+2\arrayrulewidth\relax}X|}{%
 L'amministratore ha effettuto lo use case "Effettua Login".  } \\
 \hline Success End Conditions &
 \multicolumn{4}{>{\hsize=\dimexpr 4\hsize+4\tabcolsep+2\arrayrulewidth\relax}X|}{ L'amministratore valuta una recensione. Il sistema tiene traccia di tale operazione.} \\
 \hline Failed End Conditions &
 \multicolumn{4}{>{\hsize=\dimexpr 4\hsize+4\tabcolsep+2\arrayrulewidth\relax}X|}{L'amministatore preme annulla. L'amministrazione valuta una recensione che è già stata valutata.} \\
 \hline Primary Actor &
  \multicolumn{4}{l|}{Amministratore} \\
 \hline Trigger & 
 \multicolumn{4}{>{\hsize=\dimexpr 4\hsize+4\tabcolsep+2\arrayrulewidth\relax}X|}{L'amministratore preme il pulsante "Recensioni" nella Homepage.} \\
\hline
%Main Scenario%
\multicolumn{5}{|>{\hsize=\dimexpr 4\hsize+4\tabcolsep+2\arrayrulewidth\relax}c|}{Main Scenario}\\\hline
\end{tabularx}
\setlength{\tabcolsep}{8pt}
\renewcommand{\arraystretch}{1.5}
    \begin{tabularx}{\textwidth}{|c|X|X|}
        Step\# & Amministratore & Sistema \\
        \hline
         1 &Preme il pulsante "Recensioni" nella schermata principale & \\
         \hline
         2 & & Mostra GestioneRecensioni\\
         \hline
         3 & Clicca sul radio button accanto ad una recensione e preme il pulsante "Conferma" &\\
         \hline
         4 & & Mostra ValutaRecensione\\
       \hline
         5 & Clicca sul pulsante Approva &\\
        \hline
        6& &Mostra RecensioneApprovataDialog\\
        \hline
    \end{tabularx}
    %Extension 1%
        \begin{tabularx}{\textwidth}{|c|X|X|}
            \hline
            \multicolumn{3}{|>{\hsize=\dimexpr 4\hsize+4\tabcolsep+2\arrayrulewidth\relax}c|}{Extension 1: l'amministatore rifiuta una recensione}\\\hline
            Step\# & Amministratore & Sistema \\
            \hline
             5 a &Preme il pulsante "Rifiuta". & \\
             \hline
             6 a & & Mostra RecensioneEliminataDialog.\\
            \hline
        \end{tabularx}
    
    %Extension 1%

\setlength{\tabcolsep}{8pt}
\renewcommand{\arraystretch}{1.5}
    \begin{tabularx}{\textwidth}{|c|X|X|}
        \hline
        \multicolumn{3}{|>{\hsize=\dimexpr 4\hsize+4\tabcolsep+2\arrayrulewidth\relax}c|}{Extension 2: l'amministatore preme annulla}\\\hline
        Step\# & Amministratore & Sistema \\
        \hline
         3/5 b &Preme il pulsante "Annulla". & \\
         \hline
         4/6 b & & Ritorna alla schermata principale e termina il caso d'uso.\\
        \hline
    \end{tabularx}

%Extension 2%
     \begin{tabularx}{\textwidth}{|c|X|X|}
        \hline
        \multicolumn{3}{|>{\hsize=\dimexpr 4\hsize+4\tabcolsep+2\arrayrulewidth\relax}c|}{Extension 3: la recensione è già stata valutata }\\\hline
         Step\# & Amministratore & Sistema \\
         \hline
          6 c  & & Mostra Fallimento Dialog \\
          \hline
          7 c & Preme Ok & \\
          \hline
          8 c &  & Ritorna alla schermata principale e termina il caso d'uso.\\
         \hline
     \end{tabularx}
\end{table}
\paragraph{Nota 1} Dato che il sistema può essere gestito da più amministratori è possibile che due di questi aprano contemporaneamente
la schermata di valutazione della stessa Recensione. In questo caso andrà a buon fine soltato la valutazione dell'amministatore
che per primo la rifiuterà o confermerà. L'altro amministratore vedrà la schermata di errore come descritto nella Extension 2. 
\end{document}
\input{Cockburn/Amministratore/Visualizza dati dei visitatori/visualizzaDati.tex}
\pagebreak
\subsection{Utente non autenticato} %finito

%Intestazione tabella%
\begin{table}
\begin{tabularx}{\textwidth}{|l|X|X|X|X|}
  \hline Use Case \#1 & \multicolumn{4} {l|}{Utente effettua Login} \\ \hline Goal in
  Context & \multicolumn{4}{>{\hsize=\dimexpr 4\hsize+4\tabcolsep+2\arrayrulewidth\relax}X|}{%
    L'utente non loggato effettua il login} \\
 \hline Preconditions & \multicolumn{4}{>{\hsize=\dimexpr 4\hsize+4\tabcolsep+2\arrayrulewidth\relax}X|}{%
 L'utente non autenticato non ha effettuto lo use case "Utente effettua Login".  } \\
 \hline Success End Conditions &
 \multicolumn{4}{>{\hsize=\dimexpr 4\hsize+4\tabcolsep+2\arrayrulewidth\relax}X|}{ Il login dell'utente va a buon fine.} \\
 \hline Failed End Conditions &
 \multicolumn{4}{>{\hsize=\dimexpr 4\hsize+4\tabcolsep+2\arrayrulewidth\relax}X|}{I dati di login sono errati oppure il server non è raggiungibile.} \\
 \hline Primary Actor &
  \multicolumn{4}{l|}{Utente non loggato} \\
 \hline Trigger & 
 \multicolumn{4}{>{\hsize=\dimexpr 4\hsize+4\tabcolsep+2\arrayrulewidth\relax}X|}{L'utente preme il pulsante 'Login' nel Navigation Drawer laterale dalla schermata 'HomePage utente non loggato'.} \\
\hline
%Main Scenario%
\multicolumn{5}{|>{\hsize=\dimexpr 4\hsize+4\tabcolsep+2\arrayrulewidth\relax}c|}{Main Scenario}\\\hline
\end{tabularx}
\setlength{\tabcolsep}{8pt}
\renewcommand{\arraystretch}{1.5}
    \begin{tabularx}{\textwidth}{|c|X|X|}
        Step\# & Utente & Sistema \\
        \hline
         1 &Compila correttamente i textField Username e Password  & \\
         \hline
         2 &Preme il pulsante "Login" dalla schermata LoginForm  & \\
         \hline
         3 &  &Mostra schermata HomePage\\
        \hline
    \end{tabularx}
    %Extension 1%
        \begin{tabularx}{\textwidth}{|c|X|X|}
            \hline
            \multicolumn{3}{|>{\hsize=\dimexpr 4\hsize+4\tabcolsep+2\arrayrulewidth\relax}c|}{Extension 1: l'utente inserisce dati errati}\\\hline
            Step\# & Utente & Sistema \\
            \hline
             1 a &  Compila erroneamente i textField Username e Password& \\
             \hline
             2 a & Preme il pulsante Login & \\
             \hline
             3 a & & Mostra schermata Login Dati Errati Dialog \\
             \hline
             4 a & Preme il tasto 'Riprova'&  \\
             \hline
             5 a & & Mostra Login e termina caso d'uso\\
             \hline        
        \end{tabularx} 
    %Extension 2%
    \begin{tabularx}{\textwidth}{|c|X|X|}
      \hline
      \multicolumn{3}{|>{\hsize=\dimexpr 4\hsize+4\tabcolsep+2\arrayrulewidth\relax}c|}{Extension 2: l'utente non compila tutti i campi}\\\hline
      Step\# & Utente & Sistema \\
      \hline
       1 b &  Non compila o compila solo un tra i textField Username e Password& \\
       \hline
       2 b & Preme il pulsante Login & \\
       \hline
       3 b & & Mostra schermata Login campi vuoti Dialog \\
       \hline
       4 b & Preme il tasto 'Riprova'&  \\
       \hline
       5 b & & Mostra Login e termina caso d'uso\\
       \hline        
  \end{tabularx}
\end{table}
\pagebreak
%Extension 3%
\begin{table}
\begin{tabularx}{\textwidth}{|c|X|X|}
  \hline
  \multicolumn{3}{|>{\hsize=\dimexpr 4\hsize+4\tabcolsep+2\arrayrulewidth\relax}c|}{Extension 3: il server risulta non raggiungibile}\\\hline
  Step\# & Utente & Sistema \\
  \hline
   3 c & & Mostra schermata Login server irraggiungibile \\
   \hline
   4 c & Preme il tasto 'Riprova'&  \\
   \hline
   5 c & & Mostra Login e termina caso d'uso\\
   \hline        
\end{tabularx} 
\end{table}


%Intestazione tabella%
\begin{table}[h!]
\caption{L'utente non loggato effettua la registazione}
\begin{tabularx}{\textwidth}{|l|X|X|X|X|}
  \hline Use Case \#2 & \multicolumn{4} {l|}{L'utente non loggato si registra alla piattaforma} \\ \hline Goal in
  Context & \multicolumn{4}{>{\hsize=\dimexpr 4\hsize+4\tabcolsep+2\arrayrulewidth\relax}X|}{%
    L'utente non loggato effettua la registrazione} \\
 \hline Preconditions & \multicolumn{4}{>{\hsize=\dimexpr 4\hsize+4\tabcolsep+2\arrayrulewidth\relax}X|}{%
 -  } \\
 \hline Success End Conditions &
 \multicolumn{4}{>{\hsize=\dimexpr 4\hsize+4\tabcolsep+2\arrayrulewidth\relax}X|}{ La registrazione dell'utente va a buon fine.} \\
 \hline Failed End Conditions &
 \multicolumn{4}{>{\hsize=\dimexpr 4\hsize+4\tabcolsep+2\arrayrulewidth\relax}X|}{Il server non è raggiungibile o l'utente immette dati non validi.} \\
 \hline Primary Actor &
  \multicolumn{4}{l|}{Utente non loggato} \\
 \hline Trigger & 
 \multicolumn{4}{>{\hsize=\dimexpr 4\hsize+4\tabcolsep+2\arrayrulewidth\relax}X|}{L'utente preme il pulsante 'Registrati' nel Navigation Drawer laterale dalla schermata 'HomePage utente non loggato'.} \\
\hline
%Main Scenario%
\multicolumn{5}{|>{\hsize=\dimexpr 4\hsize+4\tabcolsep+2\arrayrulewidth\relax}c|}{Main Scenario}\\\hline
\end{tabularx}
\setlength{\tabcolsep}{8pt}
\renewcommand{\arraystretch}{1.5}
    \begin{tabularx}{\textwidth}{|c|X|X|}
        Step\# & Utente & Sistema \\
        \hline
         1 &Compila correttamente tutti i textField ed il date picker  & \\
         \hline
         2 &Preme il pulsante "Fine" dalla schermata Registrazione  & \\
         \hline
         3 &  &Mostra schermata "Registrazione success dialog" e termina il caso d'uso\\
        \hline
    \end{tabularx}
  \end{table}
  \begin{table}[h!]
    \caption{Effettua registrazione - Estensione 1}
    %Extension 1%
        \begin{tabularx}{\textwidth}{|c|X|X|}
            \hline
            \multicolumn{3}{|>{\hsize=\dimexpr 4\hsize+4\tabcolsep+2\arrayrulewidth\relax}c|}{Extension 1: l'utente inserisce dati di un utente già registrato}\\\hline
            Step\# & Utente & Sistema \\
            \hline
             1 a &  Compila i textField inserendo i dati di un account già registrato& \\
             \hline
             2 a & Preme il pulsante "Fine" & \\
             \hline
             3 a & & Mostra schermata "Registrazione già registrato fail dialog e termina il caso d'uso\\
             \hline        
        \end{tabularx} 
      \end{table}

    %Extension 2%
    \begin{table}[h!]
    \caption{Effettura registrazione - Estensione 2}
    \begin{tabularx}{\textwidth}{|c|X|X|}
      \hline
      \multicolumn{3}{|>{\hsize=\dimexpr 4\hsize+4\tabcolsep+2\arrayrulewidth\relax}c|}{Extension 2: l'utente non compila tutti i campi o li compila in modo errato}\\\hline
      Step\# & Utente & Sistema \\
      \hline
       1 b &  Non compila o compila erroneamente i textfiel ed il datepicker& \\
       \hline
       2 b & Preme il pulsante "Fine" & \\
       \hline
       3 b & & Mostra schermata Login campi vuoti Dialog \\
       \hline
       4 b & Preme il tasto 'Riprova'&  \\
       \hline
       5 b & & Mostra Login e termina caso d'uso\\
       \hline        
  \end{tabularx}
\end{table}
\pagebreak
%Extension 3%
\begin{table}[h!]
  \caption{Effettura registrazione - Estensione 3}
\begin{tabularx}{\textwidth}{|c|X|X|}
  \hline
  \multicolumn{3}{|>{\hsize=\dimexpr 4\hsize+4\tabcolsep+2\arrayrulewidth\relax}c|}{Extension 3: il server risulta non raggiungibile}\\\hline
  Step\# & Utente & Sistema \\
  \hline
   3 c & & Mostra schermata Login server irraggiungibile \\
   \hline
   4 c & Preme il tasto 'Riprova'&  \\
   \hline
   5 c & & Mostra Login e termina caso d'uso\\
   \hline        
\end{tabularx} 
\end{table}


%Intestazione tabella%
\begin{table}[H]
    \caption{L'utente non loggato visulizza una struttura}
    \begin{tabularx}{\textwidth}{|l|X|X|X|X|}
      \hline Use Case \#3 & \multicolumn{4} {l|}{L'utente non loggato visulizza una struttura} \\ \hline Goal in
      Context & \multicolumn{4}{>{\hsize=\dimexpr 4\hsize+4\tabcolsep+2\arrayrulewidth\relax}X|}{%
        L'utente non loggato visulizza una struttura} \\
     \hline Preconditions & \multicolumn{4}{>{\hsize=\dimexpr 4\hsize+4\tabcolsep+2\arrayrulewidth\relax}X|}{%
     L'Utente ha effettuato una ricerca e si trova nella schermata "Lista Strutture" } \\
     \hline Success End Conditions &
     \multicolumn{4}{>{\hsize=\dimexpr 4\hsize+4\tabcolsep+2\arrayrulewidth\relax}X|}{ L'utente visualizza i dettagli di una struttura} \\
     \hline Failed End Conditions &
     \multicolumn{4}{>{\hsize=\dimexpr 4\hsize+4\tabcolsep+2\arrayrulewidth\relax}X|}{Il server non è raggiungibile } \\
     \hline Primary Actor &
      \multicolumn{4}{l|}{Utente non loggato} \\
     \hline Trigger & 
     \multicolumn{4}{>{\hsize=\dimexpr 4\hsize+4\tabcolsep+2\arrayrulewidth\relax}X|}{} \\
    \hline
    %Main Scenario%
    \multicolumn{5}{|>{\hsize=\dimexpr 4\hsize+4\tabcolsep+2\arrayrulewidth\relax}c|}{Main Scenario}\\\hline
    \end{tabularx}
    \setlength{\tabcolsep}{8pt}
    \renewcommand{\arraystretch}{1.5}
        \begin{tabularx}{\textwidth}{|c|X|X|}
            Step\# & Utente & Sistema \\
            \hline
             1 & L'utente clicca la Card di una struttura dalla schermata "Lista Strutture"'. & \\
             \hline
             2 && Mostra la schermata "Pagina Struttura utente non loggato" e termina caso d'uso\\
             \hline
            
        \end{tabularx}
    \end{table}
    \begin{table}[H]
    \caption{Visualizza struttura - Estensione 1}
    %Extension 1%
         \begin{tabularx}{\textwidth}{|c|X|X|}
                \hline
                \multicolumn{3}{|>{\hsize=\dimexpr 4\hsize+4\tabcolsep+2\arrayrulewidth\relax}c|}{Extension 1: il server non è raggiungibile}\\\hline
                Step\# & Utente & Sistema \\
                \hline
                 2 a &  & Mostra schermata "Connessione assente" e termina caso d'uso\\
                 \hline 
        \end{tabularx} 
\end{table}
    
       


%Intestazione tabella%
\begin{table}[H]
    \caption{L'utente  visulizza i dettagli di una recensione}
    \begin{tabularx}{\textwidth}{|l|X|X|X|X|}
      \hline 
      \rowcolor{Gray}
      Use Case \#4 & \multicolumn{4} {l|}{L'utente  visulizza i dettagli di una recensione} \\ \hline Goal in
      Context & \multicolumn{4}{>{\hsize=\dimexpr 4\hsize+4\tabcolsep+2\arrayrulewidth\relax}X|}
      {L'utente vuole visualizzare il testo completo di una recensione} \\
     \hline Preconditions & \multicolumn{4}{>{\hsize=\dimexpr 4\hsize+4\tabcolsep+2\arrayrulewidth\relax}X|}{%
      } \\
     \hline Success End Conditions &
     \multicolumn{4}{>{\hsize=\dimexpr 4\hsize+4\tabcolsep+2\arrayrulewidth\relax}X|}{ L'utente visualizza i dettagli di una recensione} \\
     \hline Failed End Conditions &
     \multicolumn{4}{>{\hsize=\dimexpr 4\hsize+4\tabcolsep+2\arrayrulewidth\relax}X|}{Il server non è raggiungibile } \\
     \hline Primary Actor &
      \multicolumn{4}{X|}{Utente} \\
     \hline Trigger & 
     \multicolumn{4}{>{\hsize=\dimexpr 4\hsize+4\tabcolsep+2\arrayrulewidth\relax}X|}{L'utente clicca sulla card di una recensione dalla schermata "Pagina Struttura"} \\
    \hline
    %Main Scenario%
    \multicolumn{5}{|>{\hsize=\dimexpr 4\hsize+4\tabcolsep+2\arrayrulewidth\relax}c|}{Main Scenario}\\
    \hline
\end{tabularx}
    \setlength{\tabcolsep}{8pt}
    \renewcommand{\arraystretch}{1.5}
        \begin{tabularx}{\textwidth}{|c|X|X|}
            
            Step\# & Utente & Sistema \\
            \hline
             1 & Clicca la Card di una recensione dalla schermata "Pagina struttura"'. & \\
             \hline
             2 & & Mostra la schermata "Pagina Recensione" e termina caso d'uso\\
             \hline
        \end{tabularx}
   \end{table}
    \begin{table}[H]
    \caption{Visualizza struttura - Estensione 1}
    %Extension 1%
    
         \begin{tabularx}{\textwidth}{|c|X|X|}
                \hline
                \rowcolor{LightGray}
                \multicolumn{3}{|>{\hsize=\dimexpr 4\hsize+4\tabcolsep+2\arrayrulewidth\relax}c|}{Extension 1: il server non è raggiungibile}\\\hline
                Step\# & Utente & Sistema \\
                \hline
                 2 a &  & Mostra schermata "Connessione assente" e termina caso d'uso\\
                 \hline 
        \end{tabularx} 
\end{table}
    
       


\subsection{Utente autenticato}

%Intestazione tabella%
\begin{table}[H]
    \caption{L'utente non loggato visulizza una struttura}
    \begin{tabularx}{\textwidth}{|l|X|X|X|X|}
      \hline Use Case \#3 & \multicolumn{4} {l|}{L'utente non loggato visulizza una struttura} \\ \hline Goal in
      Context & \multicolumn{4}{>{\hsize=\dimexpr 4\hsize+4\tabcolsep+2\arrayrulewidth\relax}X|}{%
        L'utente non loggato visulizza una struttura} \\
     \hline Preconditions & \multicolumn{4}{>{\hsize=\dimexpr 4\hsize+4\tabcolsep+2\arrayrulewidth\relax}X|}{%
     L'Utente ha effettuato una ricerca e si trova nella schermata "Lista Strutture" } \\
     \hline Success End Conditions &
     \multicolumn{4}{>{\hsize=\dimexpr 4\hsize+4\tabcolsep+2\arrayrulewidth\relax}X|}{ L'utente visualizza i dettagli di una struttura} \\
     \hline Failed End Conditions &
     \multicolumn{4}{>{\hsize=\dimexpr 4\hsize+4\tabcolsep+2\arrayrulewidth\relax}X|}{Il server non è raggiungibile } \\
     \hline Primary Actor &
      \multicolumn{4}{l|}{Utente non loggato} \\
     \hline Trigger & 
     \multicolumn{4}{>{\hsize=\dimexpr 4\hsize+4\tabcolsep+2\arrayrulewidth\relax}X|}{} \\
    \hline
    %Main Scenario%
    \multicolumn{5}{|>{\hsize=\dimexpr 4\hsize+4\tabcolsep+2\arrayrulewidth\relax}c|}{Main Scenario}\\\hline
    \end{tabularx}
    \setlength{\tabcolsep}{8pt}
    \renewcommand{\arraystretch}{1.5}
        \begin{tabularx}{\textwidth}{|c|X|X|}
            Step\# & Utente & Sistema \\
            \hline
             1 & L'utente clicca la Card di una struttura dalla schermata "Lista Strutture"'. & \\
             \hline
             2 && Mostra la schermata "Pagina Struttura utente non loggato" e termina caso d'uso\\
             \hline
            
        \end{tabularx}
    \end{table}
    \begin{table}[H]
    \caption{Visualizza struttura - Estensione 1}
    %Extension 1%
         \begin{tabularx}{\textwidth}{|c|X|X|}
                \hline
                \multicolumn{3}{|>{\hsize=\dimexpr 4\hsize+4\tabcolsep+2\arrayrulewidth\relax}c|}{Extension 1: il server non è raggiungibile}\\\hline
                Step\# & Utente & Sistema \\
                \hline
                 2 a &  & Mostra schermata "Connessione assente" e termina caso d'uso\\
                 \hline 
        \end{tabularx} 
\end{table}
    
       


%Intestazione tabella%
\begin{table}[H]
    \caption{L'utente aggiunge una recensione}
    \begin{tabularx}{\textwidth}{|l|X|X|X|X|}
        \rowcolor{Gray}
      \hline Use Case \#5 & \multicolumn{4} {l|}{L'utente aggiunge una recensione} \\ \hline Goal in
      Context & \multicolumn{4}{>{\hsize=\dimexpr 4\hsize+4\tabcolsep+2\arrayrulewidth\relax}X|}{%
        L'utente vuole aggiungere una recensione} \\
     \hline Preconditions & \multicolumn{4}{>{\hsize=\dimexpr 4\hsize+4\tabcolsep+2\arrayrulewidth\relax}X|}{%
      } \\
     \hline Success End Conditions &
     \multicolumn{4}{>{\hsize=\dimexpr 4\hsize+4\tabcolsep+2\arrayrulewidth\relax}X|}{ L'utente aggiunge con successo una recensione ed il sistema tiene traccia di tale operazione} \\
     \hline Failed End Conditions &
     \multicolumn{4}{>{\hsize=\dimexpr 4\hsize+4\tabcolsep+2\arrayrulewidth\relax}X|}{Il server non è raggiungibile, l'utente non compila tutti i campi} \\
     \hline Primary Actor &
      \multicolumn{4}{l|}{Utente Autenticato} \\
     \hline Trigger & 
     \multicolumn{4}{>{\hsize=\dimexpr 4\hsize+4\tabcolsep+2\arrayrulewidth\relax}X|}{L'utente clicca sul \gls{Floating Action Button} e preme su "Aggiungi recensione" dalla schermata "Pagina Struttura"} \\
    \hline
    %Main Scenario%
    \multicolumn{5}{|>{\hsize=\dimexpr 4\hsize+4\tabcolsep+2\arrayrulewidth\relax}c|}{Main Scenario}\\\hline
    \end{tabularx}
    \setlength{\tabcolsep}{8pt}
    \renewcommand{\arraystretch}{1.5}
        \begin{tabularx}{\textwidth}{|c|X|X|}
            Step\# & Utente & Sistema \\
            \hline
             1 &  & Mostra schermata Aggiungi Recensione\\
             \hline
             2 &Compila correttamente tutti i campi della schermata& \\
             \hline          
             3 &Preme il tasto "Aggiungi"& \\
             \hline
             4 && Mostra la schermata "Aggiungi recensione success Dialog" e termina il caso d'uso\\
             \hline      
        \end{tabularx}
    \end{table}
    
    %Extension 1%
    \begin{table}[H]
    \caption{Aggiungi recensione- Estensione 1}
         \begin{tabularx}{\textwidth}{|c|X|X|}
                \hline
                \rowcolor{LightGray}
                \multicolumn{3}{|>{\hsize=\dimexpr 4\hsize+4\tabcolsep+2\arrayrulewidth\relax}c|}{Extension 1: il server non è raggiungibile}\\\hline
                Step\# & Utente & Sistema \\
                \hline
                 4 a &  & Mostra schermata "Aggiungi recensione errore dialog" e termina caso d'uso\\
                 \hline 
        \end{tabularx} 
    \end{table}
    %Extension 2%
    \begin{table}[H]
        \caption{Aggiungi recensione- Estensione 2}
             \begin{tabularx}{\textwidth}{|c|X|X|}
                    \hline
                    \rowcolor{LightGray}
                    \multicolumn{3}{|>{\hsize=\dimexpr 4\hsize+4\tabcolsep+2\arrayrulewidth\relax}c|}{Extension 1: l'utente non compila uno o più campi}\\\hline
                    Step\# & Utente & Sistema \\
                    \hline
                     1 b & Compila il form tralasciando uno o più campi & \\
                     \hline 
                     2 b & Preme il tasto "Aggiungi" & \\
                     \hline 
                     3 b &  & Mostra schermata "Aggiungi recensione empty dialog" e termina caso d'uso \\
                     \hline 
            \end{tabularx} 
        \end{table}
    
       

\pagebreak
%%%%% ===============================================================================
\section{Mockup}

The redaction of the thesis has to be carried on by the candidate indipendentely. A dissertation type thesis has the structure of a scientific article where it is required to derive, from the international literature, the most recent developments on the topic of interest, it is required to synthsize them, present them in an omogenous way, and finally compare the different approaches highlighting pros and cons of each of them. A sperimental type thesis has the structure of a scientific report, it faces a specific problem, typically within a more wide project of interest forthe supervisor, proposing a solution that is innovative if compared to the state of the art. A sperimental thesis also includes a validation of the proposed solution, made by means of experimental measuraments and/or numerical simulations.

%%%%% ===============================================================================
\section{Glossario}
\textbf{Navigation Drawer} = Menù a tendina laterale\\
\textbf{Floating Action Button (FAB)} = Bottone
\begin{figure}[H]
    \caption{In evidenza un Navigation Drawer ed un Floating Action Button}
    \includegraphics[scale=0.4]{Figures/DrawerUtenteL.png}
    \includegraphics[scale=0.4]{Figures/HomePageUtenteL.png} 
\end{figure}

%%%%% ===============================================================================


% ===========================================================================
%
%		FEDERICO II THESIS TEMPLATE - ENGLISH
%  					* an example of Chapter 1: information about the discussion of the thesis
%	 
% 		AUTHOR:  		Antonio Esposito (antonio.esposito103@studenti.unina.it)
%		LAST UPDATED:	2017/06/20
%
% ===========================================================================

\chapter{Modello di Dominio}

%%%%% ===============================================================================
%\section{Classi, oggetti e relazioni di analisi}
%Qua vanno messi i class diagram
%%%%% ===============================================================================
\section{Diagrammi di sequenza di analisi}
Si riportano di seguito i diagrammi di sequenza di analisi dei casi d'uso.
\includepdf[pages={1-6}]{SequenceAnalisi/diagrammi.pdf}

\pagebreak
%%%%% ===============================================================================
\section{Diagrammi di stato di attività}
Sono riportati di seguito i diagrammi di stato di alcuni casi d'uso.
\begin{figure}[h!]
    \includegraphics[width=\textwidth]{SequenceAnalisi/7.png}
\end{figure}
\begin{figure}[h!]
    \includegraphics[width=\textwidth]{SequenceAnalisi/8.png}
\end{figure}
\pagebreak
%%%%% ===============================================================================
\section{Diagrammi di attività}
Sono riportati di seguito i diagrammi di attività di alcuni casi d'uso.
\begin{figure}[h!]
    \includegraphics[width=\textwidth]{SequenceAnalisi/9.png}
\end{figure}
\begin{figure}[h!]
    \includegraphics[width=\textwidth]{SequenceAnalisi/10.png}
\end{figure}
\begin{figure}[h!]
    \includegraphics[width=\textwidth]{SequenceAnalisi/11.png}
\end{figure}

%%%%% ===============================================================================

% ===========================================================================
%
%		FEDERICO II THESIS TEMPLATE - ENGLISH
%  					* an example of Chapter 1: information about the discussion of the thesis
%	 
% 		AUTHOR:  		Antonio Esposito (antonio.esposito103@studenti.unina.it)
%		LAST UPDATED:	2017/06/20
%
% ===========================================================================

\chapter{Impegno risorse e pianificazione dell'attività}

Si riporta di seguito il diagramma di Gantt con una proposta di impiego di risorse e di pianificazione
dell'attività.

%%%%% ===============================================================================
\section{Diagramma di Gantt}
\includepdf{DiagammaDiGantt.pdf}
\includepdf{solo tabella.pdf}


\part{Documento di Design del sistema}
% ===========================================================================
%
%		FEDERICO II THESIS TEMPLATE - ENGLISH
%  					* an example of Chapter 2: mathematical text
%	 
% 		AUTHOR:  		Antonio Esposito (antonio.esposito103@studenti.unina.it)
%		LAST UPDATED:	2017/06/20
%
% ===========================================================================

\chapter{Design}
\section{Descrizione del dominio}
In seguito ad un'analisi dei requisiti del sistema si è resa evidente l'esistenza di alcune entità
principali:
\begin{itemize}
    \item gli \textit{Utenti} che possono registrarsi, visualizzare delle \textit{Strutture}
    e scrivere delle recensioni
    \item degli \textit{Amministratori} i quali possono gestire i dati dei visitatori e le loro recensioni
\end{itemize}
%%%%% ===============================================================================
\section{Analisi dell'architettura}
Il sistema si basa principalmente su due pattern architetturali: Client-Server e Model-View-Controller.\\
Il sistema infatti è composto da diversi client che tramite internet interagiscono con un Server remoto.
\begin{center}
    \begin{figure}[H]
        \includegraphics[width=\textwidth]{Figures/Architettura client server.png}
        \caption{Architettura client server}
    \end{figure}
\end{center}
\subsection{Model-View-Presenter}
Per quanto riguarda l'architettura interna dei client si è deciso di utilizzare il modello Model-View-Presenter (variante Passive View)
favorendo la separazione dei componenti software del sistema e separandone le responsabilità.
In questo modo si riesce ad ottenere alta coesione e basso accoppiamento, ottenendo una buona manutenibilà.\\
A favore di ciò il progetto è diviso in tre sottocartelle principali: Model, View e Presenter.\\
\\
Il modello MVP è un'evoluzione del modello Model View Controller ed è usato principalmente per creare user interfaces 
ed in particolare nella variante Passive View:
la componente \textbf{View} è completamente passiva e statica deputata soltanto alla visualizzazione dei dati.\\
la componente \textbf{Model} ha la responsabilità di fornire i metodi per accedere
ai dati utili all'applicazione, ha conoscenza del dominio applicativo ed è indipendente
dagli altri sottosistemi.\\
la componente \textbf{Presenter} fa da "middle man" tra la View ed il Model e si occupa di prendere i dati dal model, 
fornirli alla view e gestire gli input dell'utente
\begin{center}
    \begin{figure}[H]
        \includegraphics[width=\textwidth]{Figures/MVP client.png}
        \caption{Architettura Model-View-Controller}
    \end{figure}
\end{center}
\section{Diagramma delle classi di design}
\phantomsection
\addcontentsline{toc}{subsection}{Applicativo Desktop}
\includepdf{Class Diagram/Design/Desktop.pdf}
\pagebreak
\includepdf{Class Diagram/Design/Mobile.pdf}
\addcontentsline{toc}{subsection}{Applicativo Mobile}
%------------------------------------
\section{CRC  Cards}
Di seguitop sono riportate tutte le Class Responsability Collaboration(CRC) del prodotto.

\subsection{Entità}
    
%Intestazione tabella%
\setcounter{table}{0}
\begin{table}[H]
    \centering
    %\caption{AdminDAO} 
    \begin{tabular}{||   l  ||  c   ||}
        \hline
        \rowcolor{Gray}
        Nome Classe & Recensione\\
        \hline
        Superclassi  &  - \\
        \hline
        Sottoclassi & - \\
        \hline
         Responsabilità & Collaboratore \\
         \hline
          Rappresenta l'entità Recensione & Struttura, Utente \\
         \hline
    \end{tabular}
    %Continua alla pagina seguente
\end{table}

    
%Intestazione tabella%
\setcounter{table}{0}
\begin{table}[H]
    \centering
    %\caption{AdminDAO} 
    \begin{tabular}{||   l  ||  c   ||}
        \hline
        Nome Classe & Struttura\\
        \hline
        Superclassi  &  - \\
        \hline
        Sottoclassi & - \\
        \hline
        \hline
         Responsabilità & Collaboratore \\
         \hline
          Rappresentare l'entità Struttura & - \\
         \hline
    \end{tabular}
    %Continua alla pagina seguente
\end{table}

    
       
 
    
%Intestazione tabella%
\setcounter{table}{0}
\begin{table}[H]
    \centering
    %\caption{AdminDAO} 
    \begin{tabular}{||   l  ||  c   ||}
        \hline
        \rowcolor{Gray}
        Nome Classe & Utente\\
        \hline
        Superclassi  &  - \\
        \hline
        Sottoclassi & - \\
        \hline
        \hline
         Responsabilità & Collaboratore \\
         \hline
          Rappresentare l'entità Utente & - \\
         \hline
    \end{tabular}
    %Continua alla pagina seguente
\end{table}

    
       

    
%Intestazione tabella%
\setcounter{table}{0}
\begin{table}[H]
    \centering
    %\caption{AdminDAO} 
    \begin{tabular}{||   l  ||  c   ||}
        \rowcolor{Gray}
        \hline
        Nome Classe & Admin\\
        \hline
        Superclasse  &  - \\
        \hline
        Sottoclassi & - \\
        \hline
        \hline
         Responsabilità & Collaboratore \\
         \hline
          Rappresentare l'entità Amministratore & - \\
         \hline
    \end{tabular}
    %Continua alla pagina seguente
\end{table}

    
       

    
%Intestazione tabella%
\setcounter{table}{0}
\begin{table}[H]
    \centering
    %\caption{AdminDAO} 
    \begin{tabular}{||   l  ||  c   ||}
        \hline
        \rowcolor{Gray}
        \textbf{Nome Classe} & Filtri\\
        \hline
        Superclassi  &  - \\
        \hline
        \textbf{Sottoclassi} & - \\
        \hline
         \textbf{Responsabilità} & \textbf{Collaboratore} \\
         \hline
          Rappresenta l'entità Filtro applicabile in una ricerca & - \\
         \hline
    \end{tabular}
    %Continua alla pagina seguente
\end{table}

\subsection{Applicativo Desktop} 
    
%Intestazione tabella%
\setcounter{table}{0}
\begin{table}[H]
    \centering
    \begin{tabular}{||   l  ||  c   ||}
        \rowcolor{Gray}
        \hline
        \textbf{Nome Classe} & AdminDAO\\
        \hline
        \textbf{Superclasse}  &  - \\
        \hline
        \textbf{Sottoclassi} & - \\
        \hline
        \hline
         \textbf{Responsabilità} & \textbf{Collaboratore} \\
         \hline
          Effettuare le operazioni CRUD per l'entità Admin & - \\
         \hline
    \end{tabular}
    
    %Continua alla pagina seguente
\end{table}

    
       

    
%Intestazione tabella%
\setcounter{table}{0}
\begin{table}[H]
    \centering
    %\caption{AdminDAO}
    \begin{tabular}{||   l  ||  c   ||}
        \hline
        Nome Classe & UtenteDAO\\
        \hline
        Superclassi  &  - \\
        \hline
        Sottoclassi & - \\
        \hline
        \hline
         Responsabilità & Collaboratore \\
         \hline
          Effettuare le operazioni CRUD per l'entità Utente & - \\
         \hline
    \end{tabular}
    %Continua alla pagina seguente
\end{table}

    
%Intestazione tabella%
\setcounter{table}{0}
\begin{table}[H]
    \centering
    %\caption{AdminDAO} 
    \begin{tabular}{||   l  ||  c   ||}
        \rowcolor{Gray}
        \hline
        \textbf{Nome Classe} & RecensioneDAO\\
        \hline
        \textbf{Superclasse}  &  - \\
        \hline
        \textbf{Sottoclassi} & - \\
        \hline
        \hline
         \textbf{Responsabilità} & \textbf{Collaboratore} \\
         \hline
          Effettuare le operazioni \gls{CRUD} per l'entità Recensione & - \\
         \hline
    \end{tabular}
    %Continua alla pagina seguente
\end{table}

    
       

    
%Intestazione tabella%
\setcounter{table}{0}
\begin{table}[H]
    \centering
    %\caption{AdminDAO} 
    \begin{tabular}{||   l  ||  c   ||}
        \rowcolor{Gray}
        \hline
        \textbf{Nome Classe} & LoginPresenter\\
        \hline
        \textbf{Superclasse}  &  - \\
        \hline
        \textbf{Sottoclassi} & - \\
        \hline
        \hline
         \textbf{Responsabilità} & \textbf{Collaboratore} \\
         \hline
          Gestire l'interazione per il login & AdminDAO, Admin \\
         \hline
    \end{tabular}
    %Continua alla pagina seguente
\end{table}


       

    
%Intestazione tabella%
\setcounter{table}{0}
\begin{table}[H]
    \centering
    %\caption{AdminDAO} 
    \begin{tabular}{||   l  ||  c   ||}
        \hline
        \rowcolor{Gray}
        \textbf{Nome Classe} & SideMenuPresenter\\
        \hline
        \textbf{Superclasse}  &  - \\
        \hline
        \textbf{Sottoclassi} & - \\
        \hline
        \hline
         \textbf{Responsabilità} & \textbf{Collaboratore} \\
         \hline
          Gestire l'interazione con la SideBar & Admin \\
         \hline
    \end{tabular}
    %Continua alla pagina seguente
\end{table}

    
       

    
%Intestazione tabella%
\setcounter{table}{0}
\begin{table}[H]
    \centering
    %\caption{AdminDAO} 
    \begin{tabular}{||   l  ||  c   ||}
        \rowcolor{Gray}
        \rowcolor{Gray}
        \hline
        \textbf{Nome Classe} & ModificaUtentePresenter\\
        \hline
        \textbf{Superclasse}  &  - \\
        \hline
        \textbf{Sottoclassi} & - \\
        \hline
        \hline
         \textbf{Responsabilità} & \textbf{Collaboratore} \\
         \hline
          Gestire l'interazione per la modifica\newline dei dati di accesso di un utente & Utente, UtenteDAO \\
         \hline
    \end{tabular}
    %Continua alla pagina seguente
\end{table}

    
       

    
%Intestazione tabella%
\setcounter{table}{0}
\begin{table}[H]
    \centering
    %\caption{AdminDAO} 
    \begin{tabular}{||   l  ||  c   ||}
        \rowcolor{Gray}
        \hline
        \textbf{Nome Classe} & RecensioniPresenter\\
        \hline
        \textbf{Superclasse}  &  - \\
        \hline
        \textbf{Sottoclassi} & - \\
        \hline
        \hline
         \textbf{Responsabilità} & \textbf{Collaboratore} \\
         \hline
          Riempire la tabella con le recensioni in attesa di approvazione & RecensioneDAO \\
         \hline
    \end{tabular}
    %Continua alla pagina seguente
\end{table}

    
       

    
%Intestazione tabella%
\setcounter{table}{0}
\begin{table}[H]
    \centering
    %\caption{AdminDAO} 
    \begin{tabular}{||   l  ||  c   ||}
        \hline
        \rowcolor{Gray}
        \textbf{Nome Classe} & UtentiPresenter\\
        \hline
        \textbf{Superclasse}  &  - \\
        \hline
        \textbf{Sottoclassi} & - \\
        \hline
        \hline
         \textbf{Responsabilità} & \textbf{Collaboratore} \\
         \hline
         Riempire la tabella con gli utenti registrati alla piattaforma & UtenteDAO \\
         \hline
    \end{tabular}
    %Continua alla pagina seguente
\end{table}

    
       

    
%Intestazione tabella%
\setcounter{table}{0}
\begin{table}[H]
    \centering
    %\caption{AdminDAO} 
    \begin{tabularx}{\textwidth}{||   X  ||  c   ||}
        \hline
        \rowcolor{Gray}
        \textbf{Nome Classe} & VisualizzaRecensionePresenter\\
        \hline
        \textbf{Superclasse}  &  - \\
        \hline
        \textbf{Sottoclassi} & - \\
        \hline
        \hline
         \textbf{Responsabilità} & \textbf{Collaboratore} \\
         \hline
          Gestire l'interazione per la moderazione\newline di una recensione da parte di un admin & RecensioneDAO, Recensione \\
         \hline
    \end{tabularx}
    %Continua alla pagina seguente
\end{table}

    
       

    
%Intestazione tabella%
\setcounter{table}{0}
\begin{table}[H]
    \centering
    %\caption{AdminDAO} 
    \begin{tabular}{||   l  ||  c   ||}
        \hline
        \rowcolor{Gray}
        \textbf{Nome Classe} & VisualizzaUtentePresenter\\
        \hline
        \textbf{Superclasse}  &  - \\
        \hline
        \textbf{Sottoclassi} & - \\
        \hline
        \hline
         \textbf{Responsabilità} & \textbf{Collaboratore} \\
         \hline
          Mostrare i dati completi di un utente\newline e permetterne l'eliminazione & UtenteDAO, Utente \\
         \hline
    \end{tabular}
    %Continua alla pagina seguente
\end{table}

    
       

    
%Intestazione tabella%
\setcounter{table}{0}
\begin{table}[H]
    \centering
    %\caption{AdminDAO} 
    \begin{tabularx}{\textwidth}{||   X  ||  c   ||}
        \rowcolor{Gray}
        \hline
        \textbf{Nome Classe} & PasswordUtils\\
        \hline
        \textbf{Superclasse}  &  - \\
        \hline
        \textbf{Sottoclassi} & - \\
        \hline
        \hline
         \textbf{Responsabilità} & \textbf{Collaboratore} \\
         \hline
          Fornire funzionalità per il sistema di cifratura\newline delle password tramite \gls{Sale}   & - \\
         \hline
    \end{tabularx}
    %Continua alla pagina seguente
\end{table}

    
       

    
%Intestazione tabella%
\setcounter{table}{0}
\begin{table}[H]
    \centering
    %\caption{AdminDAO} 
    \begin{tabular}{||   l  ||  c   ||}
        \rowcolor{Gray}
        \hline
        \textbf{Nome Classe} & MainApp\\
        \hline
        \textbf{Superclasse}  &  Application \\
        \hline
        \textbf{Sottoclassi} & - \\
        \hline
        \hline
         \textbf{Responsabilità} & \textbf{Collaboratore} \\
         \hline
          Avviare con i giusti parametri la finestra con la prima schermata & - \\
         \hline
    \end{tabular}
    %Continua alla pagina seguente
\end{table}

    
       

    \pagebreak
\subsection{Applicativo Mobile}
    
%Intestazione tabella%
\setcounter{table}{0}
\begin{table}[H]
    \centering
    %\caption{AdminDAO}
    \begin{tabular}{||   l  ||  c   ||}
        \hline
        Nome Classe & UtenteDAO\\
        \hline
        Superclassi  &  - \\
        \hline
        Sottoclassi & - \\
        \hline
        \hline
         Responsabilità & Collaboratore \\
         \hline
          Effettuare le operazioni CRUD per l'entità Utente & - \\
         \hline
    \end{tabular}
    %Continua alla pagina seguente
\end{table}

    
%Intestazione tabella%
\setcounter{table}{0}
\begin{table}[H]
    \centering
    %\caption{AdminDAO} 
    \begin{tabular}{||   l  ||  c   ||}
        \rowcolor{Gray}
        \hline
        \textbf{Nome Classe} & RecensioneDAO\\
        \hline
        \textbf{Superclasse}  &  - \\
        \hline
        \textbf{Sottoclassi} & - \\
        \hline
        \hline
         \textbf{Responsabilità} & \textbf{Collaboratore} \\
         \hline
          Effettuare le operazioni \gls{CRUD} per l'entità Recensione & - \\
         \hline
    \end{tabular}
    %Continua alla pagina seguente
\end{table}

    
       

    
%Intestazione tabella%
\setcounter{table}{0}
\begin{table}[H]
    \centering
    %\caption{AdminDAO} 
    \begin{tabular}{||   l  ||  c   ||}
        \hline
        \rowcolor{Gray}
        \textbf{Nome Classe} & StrutturaDAO\\
        \hline
        Superclassi  &  - \\
        \hline
        \textbf{Sottoclassi} & - \\
        \hline
         \textbf{Responsabilità} & \textbf{Collaboratore} \\
         \hline
          Effettuare le operazioni \gls{CRUD} per l'entità Struttura & - \\
         \hline
    \end{tabular}
    %Continua alla pagina seguente
\end{table}

    
%Intestazione tabella%
\setcounter{table}{0}
\begin{table}[H]
    \centering
    %\caption{AdminDAO} 
    \begin{tabular}{||   l  ||  c   ||}
        \hline
        \rowcolor{Gray}
        \textbf{Nome Classe} & RegistrazioneFragment\\
        \hline
        Superclassi  &  - \\
        \hline
        \textbf{Sottoclassi} & - \\
        \hline
        \hline
         \textbf{Responsabilità} & \textbf{Collaboratore} \\
         \hline
          Mostra il form di registrazione di un utente & - \\
         \hline
          Gestisce l'operazione di registrazione di un utente & UtenteDAO, Utente \\
         \hline
    \end{tabular}
    %Continua alla pagina seguente
\end{table}

    
%Intestazione tabella%
\setcounter{table}{0}
\begin{table}[H]
    \centering
    %\caption{AdminDAO} 
    \begin{tabular}{||   l  ||  c   ||}
        \hline
        \rowcolor{Gray}
        \textbf{Nome Classe} & LoginFragment\\
        \hline
        Superclassi  &  - \\
        \hline
        \textbf{Sottoclassi} & - \\
        \hline
        \hline
         \textbf{Responsabilità} & \textbf{Collaboratore} \\
         \hline
           Gestisce l'accesso, tramite credenziali, di un utente & UtenteDAO \\
         \hline
    \end{tabular}
    %Continua alla pagina seguente
\end{table}

    
%Intestazione tabella%
\setcounter{table}{0}
\begin{table}[H]
    \centering
    %\caption{AdminDAO} 
    \begin{tabularx}{\textwidth}{||   X  ||  c   ||}
        \hline
        \rowcolor{Gray}
        \textbf{Nome Classe} & AggiungiRecensioneFragment\\
        \hline
        Superclassi  &  - \\
        \hline
        \textbf{Sottoclassi} & - \\
        \hline
        \hline
         \textbf{Responsabilità} & \textbf{Collaboratore} \\
         \hline
          Mostra il form per l'aggiunta di una recensione &  \\
          \hline
          Gestisce l'aggiunta di una recensione & RecensioneDAO \\
         \hline
    \end{tabularx}
    %Continua alla pagina seguente
\end{table}

    
%Intestazione tabella%
\setcounter{table}{0}
\begin{table}[H]
    \centering
    %\caption{AdminDAO} 
    \begin{tabularx}{\textwidth}{||   X  ||  c   ||}
        \hline
        \rowcolor{Gray}
        \textbf{Nome Classe} & ConnessioneAssenteFragment\\
        \hline
        Superclassi  &  - \\
        \hline
        \textbf{Sottoclassi} & - \\
        \hline
         \textbf{Responsabilità} & \textbf{Collaboratore} \\
         \hline
           Mostra una schermata che avvisa l'utente dell'impossibilità di connettersi al server.  & - \\
         \hline
    \end{tabularx}
    %Continua alla pagina seguente
\end{table}

    
%Intestazione tabella%
\setcounter{table}{0}
\begin{table}[H]
    \centering
    %\caption{AdminDAO} 
    \begin{tabularx}{\textwidth}{||   X  ||  c   ||}
        \hline
        \rowcolor{Gray}
        \textbf{Nome Classe} & ListaRecensioniRecyclerViewAdapter\\
        \hline
        Superclassi  &  - \\
        \hline
        \textbf{Sottoclassi} & - \\
        \hline
         \textbf{Responsabilità} & \textbf{Collaboratore} \\
         \hline
          Gestisce la lista delle strutture mostrate & Struttura \\
         \hline
          Gestisce il click sulle strutture della lista & Struttura \\
         \hline
    \end{tabularx}
    %Continua alla pagina seguente
\end{table}

    
%Intestazione tabella%
\setcounter{table}{0}
\begin{table}[H]
    \centering
    %\caption{AdminDAO} 
    \begin{tabular}{||   l  ||  c   ||}
        \hline
        \rowcolor{Gray}
        \textbf{Nome Classe} & RecensioneDAO\\
        \hline
        Superclassi  &  - \\
        \hline
        \textbf{Sottoclassi} & - \\
        \hline
         \textbf{Responsabilità} & \textbf{Collaboratore} \\
         \hline
          Mostra la lista delle strutture risultanti da una ricerca & Strutture \\
         \hline
    \end{tabular}
    %Continua alla pagina seguente
\end{table}

    
%Intestazione tabella%
\setcounter{table}{0}
\begin{table}[H]
    \centering
    %\caption{AdminDAO} 
    \begin{tabularx}{\textwidth}{||   X  ||  c   ||}
        \hline
        \rowcolor{Gray}
        \textbf{Nome Classe} & DettagliStrutturaFragment\\
        \hline
        Superclassi  &  - \\
        \hline
        \textbf{Sottoclassi} & - \\
        \hline
        \hline
         \textbf{Responsabilità} & \textbf{Collaboratore} \\
         \hline
          Permette la visualizzazione dei dati di una struttura & Struttura \\
         \hline
          Permette la visualizzazione delle Recensioni di una struttura & Recensione, RecensioneDAO, StrutturaDAO \\
         \hline
    \end{tabularx}
    %Continua alla pagina seguente
\end{table}

    
%Intestazione tabella%
\setcounter{table}{0}
\begin{table}[H]
    \centering
    %\caption{AdminDAO} 
    \begin{tabularx}{\textwidth}{||   X  ||  c   ||}
        \rowcolor{Gray}
        \hline
        \textbf{Nome Classe} & PasswordUtils\\
        \hline
        \textbf{Superclasse}  &  - \\
        \hline
        \textbf{Sottoclassi} & - \\
        \hline
        \hline
         \textbf{Responsabilità} & \textbf{Collaboratore} \\
         \hline
          Fornire funzionalità per il sistema di cifratura\newline delle password tramite \gls{Sale}   & - \\
         \hline
    \end{tabularx}
    %Continua alla pagina seguente
\end{table}

    
       

    
%Intestazione tabella%
\setcounter{table}{0}
\begin{table}[H]
    \centering
    %\caption{AdminDAO} 
    \begin{tabularx}{\textwidth}{||   X  ||  c   ||}
        \hline
        \rowcolor{Gray}
        \textbf{Nome Classe} & NessunaStrutturaTrovata\\
        \hline
        Superclassi  &  - \\
        \hline
        \textbf{Sottoclassi} & - \\
        \hline
        \hline
         \textbf{Responsabilità} & \textbf{Collaboratore} \\
         \hline
           Mostra una schermata apposita nel caso in cui una ricerca
           non abbia nessun risultato da visualizzare & - \\
         \hline
    \end{tabularx}
    %Continua alla pagina seguente
\end{table}

    \input{CRC/Mobile/VisualizzaRecensioneFragment-CRC.tex}
    
%Intestazione tabella%
\setcounter{table}{0}
\begin{table}[H]
    \centering
    %\caption{AdminDAO} 
    \begin{tabular}{||   l  ||  c   ||}
        \hline
        \rowcolor{Gray}
        \textbf{Nome Classe} & FiltriFragment\\
        \hline
        Superclassi  &  - \\
        \hline
        \textbf{Sottoclassi} & - \\
        \hline
         \textbf{Responsabilità} & \textbf{Collaboratore} \\
         \hline
          Gestisce l'applicazione dei flitri alla ricerca delle strutture & StrutturaDAO \\
         \hline
    \end{tabular}
    %Continua alla pagina seguente
\end{table}

    
%Intestazione tabella%
\setcounter{table}{0}
\begin{table}[H]
    \centering
    %\caption{AdminDAO} 
    \begin{tabularx}{\textwidth}{||   X  ||  c   ||}
        \hline
        \rowcolor{Gray}
        \textbf{Nome Classe} & FiltriRecensioniDialog\\
        \hline
        Superclassi                                             &  - \\
        \hline
        \textbf{Sottoclassi}                                    & - \\
        \hline
        \hline
        \textbf{Responsabilità}                                 & \textbf{Collaboratore} \\
        \hline
        Gestisce l'applicazione di filtri alla lista delle recensioni di una Struttura   & - \\
        \hline
    \end{tabularx}
    %Continua alla pagina seguente
\end{table}

    
%Intestazione tabella%
\setcounter{table}{0}
\begin{table}[H]
    \centering
    %\caption{AdminDAO} 
    \begin{tabularx}{\textwidth}{||   X  ||  c   ||}
        \hline
        \rowcolor{Gray}
        \textbf{Nome Classe} & VisualizzaSuMappaDialog\\
        \hline
        Superclassi  &  - \\
        \hline
        \textbf{Sottoclassi} & - \\
        \hline
        \hline
         \textbf{Responsabilità} & \textbf{Collaboratore} \\
         \hline
         Permette di visualizzare la posizione di una Struttura sulla mappa, all'interno di un dialog& Struttura \\
         \hline
    \end{tabularx}
    %Continua alla pagina seguente
\end{table}

    
%Intestazione tabella%
\setcounter{table}{0}
\begin{table}[H]
    \centering
    %\caption{AdminDAO} 
    \begin{tabularx}{\textwidth}{||   X  ||  c   ||}
        \hline
        \rowcolor{Gray}
        \textbf{Nome Classe} & MainActivity\\
        \hline
        Superclassi  &  - \\
        \hline
        \textbf{Sottoclassi} & - \\
        \hline
        \hline
         \textbf{Responsabilità} & \textbf{Collaboratore} \\
         \hline
            Inizializza l'app quando viene avviata & - \\
         \hline
          Richiede i permessi per l'uso del gps & - \\
         \hline
          Visualizza la mappa sulla homepage & - \\
         \hline
          Visualizza la schermata di una struttura dopo averci cliccato sopra dalla mappa & StrutturaDAO \\
         \hline
    \end{tabularx}
    %Continua alla pagina seguente
\end{table}

    
%Intestazione tabella%
\setcounter{table}{0}
\begin{table}[H]
    \centering
    %\caption{AdminDAO} 
    \begin{tabular}{||   l  ||  c   ||}
        \hline
        \rowcolor{Gray}
        \textbf{Nome Classe} & CustomSupportMapFragment\\
        \hline
        Superclassi  &  - \\
        \hline
        \textbf{Sottoclassi} & - \\
        \hline
         \textbf{Responsabilità} & \textbf{Collaboratore} \\
         \hline
          Gestisce la mappa visualizzata sull'homepage & - \\
         \hline
    \end{tabular}
    %Continua alla pagina seguente
\end{table}

%------------------------------------
\section{Diagramma di stato di design}

\section{Diagramma di sequenza di design}
\includepdf[pages={-}]{Sequence/Design/sequence.pdf}

\part{Documento di Testing del sistema}
% ===========================================================================
%
%		FEDERICO II THESIS TEMPLATE - ENGLISH
%  					* an example of Chapter 3: tables, figures and software code
%	 
% 		AUTHOR:  		Antonio Esposito (antonio.esposito103@studenti.unina.it)
%		LAST UPDATED:	2017/06/20
%
% ===========================================================================

\chapter{Testing}

Il documento di testing è mirato a verificare e validare il sistema in modo che sia 
conforme alle specifiche ed ai requisiti evidenziati dai precedenti documenti.

%%%%% ===============================================================================
\section{Test Plan per il System Testing}
\subsection{Applicativo Desktop}

%Intestazione tabella%
\setcounter{table}{0}
\begin{table}[H]
    \centering
    \footnotesize
    \caption{Test Case \#1}
    \begin{tabularx}{\textwidth}{|c|X|}
        \hline
        Applicativo & Desktop\\
        \hline
        Nome & Amministratore effettua login  \\
        \hline
        Descrizione & Test che assicura il corretto funzionamento del login dell'amministratore\\
        \hline
        Note & Esiste un amministratore con le seguenti credenziali: "admin", "admin" \\
        \hline
        Stato & Passato/Fallito\\
        \hline

    \end{tabularx}
    Continua alla pagina seguente
    \setlength{\tabcolsep}{8pt}
    \renewcommand{\arraystretch}{1.5}
\end{table}
\begin{table}[H]
    \footnotesize
    \begin{tabularx}{\textwidth}{|c|X|X|X|}
        \hline
        Step\# & Input & Risultato Atteso & Risultato Ottenuto \\
        \hline
         1 & Viene aperto l'applicativo Desktop  
         & Viene visualizzata la schermata "Login" contententue due Textbox (Username e Password) 
         ed un tasto
         &Come ci si aspettava\\
          \hline
        2 & Vengono inserite delle credenziali \textbf{errate} "0000", "0000" e viene cliccato 
        il tasto "Login" 
        & Il login fallisce e viene visualizzata il dialog "Credenziali errate"
        & Come ci si aspettava\\
         \hline
         3 & Viene premuto il tasto "OK"
        & Viene mostrata nuovamente la schermata di login
        & Come ci si aspettava\\
        4 & Vengono inserite le credenziali "admin", "admin" e viene cliccato 
        il tasto "Login" 
        & Il login ha successo e viene visualizzata la schermata "
        Revisioni" 
        & Come ci si aspettava\\
         \hline
                 
    \end{tabularx}
\end{table}
    
       


%Intestazione tabella%
\begin{table}[H]
    \footnotesize
    \caption{Test Case \#2}
    \begin{tabularx}{\textwidth}{|c|X|}
        \hline
        Applicativo & Desktop\\
        \hline
        Nome & Amministratore visualizza dati di un utente  \\
        \hline
        Descrizione & Test che assicura il corretto funzionamento della visualizzazione dei dati dell'utente\\
        \hline
        Note & L'amministratore deve essere loggato, deve esistere l'utente con nickname "utente" \\
        \hline
        Stato & Passato/Fallito\\
        \hline

    \end{tabularx}
    \setlength{\tabcolsep}{8pt}
    \renewcommand{\arraystretch}{1.5}
\end{table}

\begin{table}[H]
    \footnotesize
    \begin{tabularx}{\textwidth}{|c|X|X|X|}
        \hline
        Step\# & Input & Risultato Atteso & Risultato Ottenuto \\
        \hline
         1 & Viene premuto il tasto "Visitatori" dalla schermata "Recensioni" 
         & Viene visualizzata la schermata "Visitatori" contententue una lista di visitatori
         &DA INSERIRE (Come ci si aspettava)\\
          \hline
        2 & Viene cliccata la riga contenente l'utente con nickname "utente"
        & Viene visualizzata la schermata utente contenete i tasti "Modifica", "Elimina" ed i textfild nome, nickname, data di nascita, email, recensioni rifiutate ed approvate contenenti i relativi dati.
        & DA INSERIRE\\
         \hline  
    \end{tabularx}
\end{table}
    
       

\pagebreak
%Intestazione tabella%
\begin{table}[H]
    \centering
    \footnotesize
    \caption{Test Case \#3}
    \begin{tabularx}{\textwidth}{|c|X|}
        \hline
        Applicativo & Desktop\\
        \hline
        Nome & Amministratore valuta una recensione  \\
        \hline
        Descrizione & Test che assicura il corretto funzionamento della valutazione delle recensioni\\
        \hline
        Note & L'amministatore deve essere autenticato e deve esistere una recensione scritta da "utente" per la struttura "Struttura" in data "GEN 1 2020" \\
        \hline
        Stato & Passato/Fallito\\
        \hline

    \end{tabularx}
    %Continua alla pagina seguente
    \setlength{\tabcolsep}{8pt}
    \renewcommand{\arraystretch}{1.5}
\end{table}

\begin{table}[H]
    \footnotesize
    \begin{tabularx}{\textwidth}{|c|X|X|X|}
        \hline
        Step\# & Input & Risultato Atteso & Risultato Ottenuto \\
        \hline
         1 & Viene premuto il tasto "Recensioni" dalla schermata "Visitatori" 
         & Viene visualizzata la schermata "Recensioni" contententue una lista di recensioni
         &Come ci si aspettava\\
          \hline
        2 & Viene cliccata la riga contenente la recensione scritta in data "GEN 1 2020" dall'utente "utente" sulla struttura "Struttura"
        & Viene visualizzata la schermata della recensione contenete i tasti "Acceta" e  "Rifiuta", i textFiel contenenti il titolo, il testo e l'autore della descrizione.
        Viene visualizzato, in oltre, il numero di stelline della recensione.
        & Come ci si aspettava\\
         \hline 
        3 & Viene cliccato il bottone "Accetta" o "Rifiuta"
         & Viene visualizzato il dialog che ci informa che l'azione è stata svolta con successo
         & Come ci si aspettava\\
          \hline
          4 & Viene premuto il tasto "Ok"
         & Viene visualizzata la schermata Recensioni e nella lista delle recensioni non compare più 
         la recensione scritta dall'utente "Utente" sulla struttura "Struttura" in data "GEN 1 2020"
         & Come ci si aspettava\\
          \hline      
    \end{tabularx}
\end{table}
    
       


%Intestazione tabella%
\begin{table}[H]
    \centering
    \footnotesize
    \caption{Test Case \#4}
    \begin{tabularx}{\textwidth}{|c|X|}
        \hline
        Applicativo & Desktop\\
        \hline
        Nome & Amministratore elimina visitatore  \\
        \hline
        Descrizione & Test che assicura il corretto funzionamento dell'eliminazione di un visitatore'\\
        \hline
        Note & L'amministatore deve essere autenticato e deve esistere l'utente "utente"  \\
        \hline
        Stato & Passato/Fallito\\
        \hline

    \end{tabularx}
    Continua alla pagina seguente
    \setlength{\tabcolsep}{8pt}
    \renewcommand{\arraystretch}{1.5}
\end{table}
%Step%
\begin{table}[H]
    \footnotesize
    \begin{tabularx}{\textwidth}{|c|X|X|X|}
        \hline
        Step\# & Input & Risultato Atteso & Risultato Ottenuto \\
        \hline
         1 & Viene premuto il tasto "Visitatori" dalla schermata "Recensioni" 
         & Viene visualizzata la schermata "Visitatori" contententue una lista di visitatori
         &DA INSERIRE (Come ci si aspettava)\\
          \hline
        2 & Viene cliccata la riga contenente l'utente "utente"
        & Viene visualizzata la schermata di visualizzazione dell'utente contenente i bottoni "Elimina" e "Modifica"
          Sulla schermata sono presenti anche i seguenti textfield: nome, email, nickname, data di nascita, recensioni approvate e rifiutate .
        & DA INSERIRE\\
         \hline 
        3 & Viene cliccato il bottone "Elimina"
         & Viene visualizzato il dialog che ci informa che l'utente è stato eliminato con successo
         & DA INSERIRE\\
          \hline
          4 & Viene premuto il tasto "Ok"
         & Viene visualizzata visualizzata nuovamente la schermata con la lista dei visitatori
         & DA INSERIRE\\
          \hline      
    \end{tabularx}
\end{table}
    
       


%Intestazione tabella%
\begin{table}[H]
    \centering
    \footnotesize
    \caption{Test Case \#5}
    \begin{tabularx}{\textwidth}{|c|X|}
        \hline
        Applicativo & Desktop\\
        \hline
        Nome & Amministratore modifica dati visitatore  \\
        \hline
        Descrizione & Test che assicura il corretto funzionamento della modifica dei dati di un visitatore'\\
        \hline
        Note & L'amministatore deve essere autenticato e deve esistere l'utente "utente" con nickname "user"  \\
        \hline
        Stato & Passato/Fallito\\
        \hline

    \end{tabularx}
    Continua alla pagina seguente
    \setlength{\tabcolsep}{8pt}
    \renewcommand{\arraystretch}{1.5}
\end{table}
%Step%
\begin{table}[H]
    \footnotesize
    \begin{tabularx}{\textwidth}{|c|X|X|X|}
        \hline
        Step\# & Input & Risultato Atteso & Risultato Ottenuto \\
        \hline
         1 & Viene premuto il tasto "Visitatori" dalla schermata "Recensioni" 
         & Viene visualizzata la schermata "Visitatori" contententue una lista di visitatori
         &DA INSERIRE (Come ci si aspettava)\\
          \hline
        2 & Viene cliccata la riga contenente l'utente "utente"
        & Viene visualizzata la schermata di visualizzazione dell'utente contenente i bottoni "Elimina" e "Modifica"
          Sulla schermata sono presenti anche i seguenti textfield: nome, email, nickname, data di nascita, recensioni approvate e rifiutate .
        & DA INSERIRE\\
         \hline 
        3 & Viene cliccato il bottone "Modifica"
         & Viene visualizzata la schemata di modifica dei tasti utente contenente due editText ("Nickname" e "Password") e
         due bottoni ("Annulla" e "Conferma")
         & DA INSERIRE\\
          \hline
        4 & Viene inserito il Nickname "user" e la nuova password: "pass"; viene premuto il tasto "Ok"
         & Viene visualizzata un dialog che avvisa che l'azione non ha avuto successo
         & DA INSERIRE\\
          \hline  
          5 & Viene premuto il tasto "OK"
          & Viene visualizzata nuovamente la schermata contenente le informazioni di "utente"
          & DA INSERIRE\\
          \hline      
        6 & Viene inserito il Nickname "nuovoNickname" e la nuova password: "pass"; viene premuto il tasto "Ok"
         & Viene visualizzata un dialog che avvisa che l'azione ha avuto successo
         & DA INSERIRE\\
           \hline 
           7 & Viene premuto il tasto "Ok"
           & Viene visualizzata nuovamente la schermata con le informazioni dell'utente e
           viene visualizzato il nickname ("nuovoNickname")
           & DA INSERIRE\\
             \hline                       
    \end{tabularx}
\end{table}
    
       


\pagebreak
\subsection{Applicativo Mobile}

%Intestazione tabella%
\begin{table}[H]
    \centering
    \footnotesize
    \caption{Test Case \#1}
    \begin{tabularx}{\textwidth}{|c|X|}
        \hline
        Applicativo & Mobile\\
        \hline
        Nome & Utente si registra alla piattaforma  \\
        \hline
        Descrizione & Test che assicura il corretto funzionamento della funzione di registrazione\\
        \hline
        Note &  L'utente non deve aver effettuato il login e non deve esistere nessun utente
         registrato con i seguenti dati: nickname = "nick" email = "email.email@gmail.com".\\
        \hline
        Stato & Passato/Fallito\\
        \hline

    \end{tabularx}
    %Continua alla pagina seguente
    \setlength{\tabcolsep}{8pt}
    \renewcommand{\arraystretch}{1.5}
\end{table}
%Step%
\begin{table}[H]
    \footnotesize
    \begin{tabularx}{\textwidth}{|c|X|X|X|}
        \hline
        Step\# & Input & Risultato Atteso & Risultato Ottenuto \\
        \hline
         1 & Dall'homepage viene cliccata l'iconda del menù in alto a sinistra 
         & Viene visualizzato il menù laterale sinsitra
         &DA INSERIRE (Come ci si aspettava)\\
          \hline
        2 & Viene cliccata la riga contenente "Registrati"
        & Viene visualizzata la schermata di registrazione con il bottone "Fine", 5 TextField, 1 datePicker ed una checkbox
        & DA INSERIRE\\
         \hline 
        3 & Non viene compilato nessun campo e viene premuto il tasto "Fine"
         & Viene mostrato un dialog di errore & DA INSERIRE\\
          \hline
        4 & Viene premuto il tasto "Ok"
         & Viene visualizzata nuovamente la schermata di registrazione
         & DA INSERIRE\\
          \hline 
          5 & Vengono inseriti i campi: "email.email@gmail.com", "password","Mario",
          "Rossi", "17/11/1998", "nick" e viene premuto il tasto "Fine".
         & Viene visualizzata mostrato il dialog di registrazione effettuata
         & DA INSERIRE\\
          \hline           
    \end{tabularx}
\end{table}
    
       


%Intestazione tabella%
\begin{table}[H]
    \centering
    \footnotesize
    \caption{Test Case \#2}
    \begin{tabularx}{\textwidth}{|c|X|}
        \hline
        Applicativo & Mobile\\
        \hline
        Nome & Utente accede alla piattaforma  \\
        \hline
        Descrizione & Test che assicura il corretto funzionamento della funzione di Login\\
        \hline
        Note &  L'utente non deve aver effettuato il login e deve esistere l'utente
        con nickname="nickname" e password "password"\\
        \hline
        Stato & Passato/Fallito\\
        \hline

    \end{tabularx}
    Continua alla pagina seguente
    \setlength{\tabcolsep}{8pt}
    \renewcommand{\arraystretch}{1.5}
\end{table}
%Step%
\begin{table}[H]
    \footnotesize
    \begin{tabularx}{\textwidth}{|c|X|X|X|}
        \hline
        Step\# & Input & Risultato Atteso & Risultato Ottenuto \\
        \hline
         1 & Dall'homepage viene cliccata l'iconda del menù in alto a sinistra 
         & Viene visualizzato il menù laterale sinsitra
         &DA INSERIRE (Come ci si aspettava)\\
          \hline
        2 & Viene cliccata la riga contenente "Login"
        & Viene visualizzata la schermata di Login con il bottone "Login" 2 TextField
        & DA INSERIRE\\
         \hline 
        3 & Non viene compilato nessun campo e viene premuto il tasto "Fine"
         & Viene mostrato un dialog di errore & DA INSERIRE\\
          \hline
        4 & Viene premuto il tasto "Ok"
         & Viene visualizzata nuovamente la schermata di Login
         & DA INSERIRE\\
          \hline 
          5 & Vengono inseriti i campi: "nickname" e "password" e viene premuto il tasto "Login"
         & Viene visualizzata mostrata la homepage dell'applicativo
         & DA INSERIRE\\
          \hline           
    \end{tabularx}
\end{table}
    
       


%Intestazione tabella%
\begin{table}[H]
    \centering
    \footnotesize
    \caption{Test Case \#3}
    \begin{tabularx}{\textwidth}{|c|X|}
        \hline
        Applicativo & Mobile\\
        \hline
        Nome & Utente aggiunge una recensione  \\
        \hline
        Descrizione & Test che assicura il corretto funzionamento dell'aggiunta di recensioni\\
        \hline
        Note &  L'utente deve aver effettuato il login\\
        \hline
        Stato & Passato/Fallito\\
        \hline

    \end{tabularx}
    Continua alla pagina seguente
    \setlength{\tabcolsep}{8pt}
    \renewcommand{\arraystretch}{1.5}
\end{table}
%Step%
\begin{table}[H]
    \footnotesize
    \begin{tabularx}{\textwidth}{|c|X|X|X|}
        \hline
        Step\# & Input & Risultato Atteso & Risultato Ottenuto \\
        \hline
         1 & Viene aperta la schermata di visualizzazione di una struttura
         & Viene visualizzato una schermata contenente tutti i dettagli della struttura
         &DA INSERIRE (Come ci si aspettava)\\
          \hline
        2 & Viene cliccato il pulsante "Aggiungi Recensione"
        & Viene visualizzata la schermata di aggiunta recensione
        & DA INSERIRE\\
         \hline 
        3 & Non viene compilato nessun campo e viene premuto il tasto "Fine"
         & Viene mostrato un dialog di errore & DA INSERIRE\\
          \hline
        4 & Viene premuto il tasto "Ok"
         & Viene visualizzata nuovamente la schermata di Login
         & DA INSERIRE\\
          \hline 
          5 & Vengono inseriti i campi: "titolo", "testo", viene cliccata la prima stellina a sinistra e viene premuto il tasto "Aggiungi"
         & Viene visualizzato dialog che notifica l'avvenuta operazione.
         & DA INSERIRE\\
          \hline 
          6 & Preme "Ok"
          & Viene visualizzato nuovamente la schermata della struttura
          & DA INSERIRE\\
           \hline                     
    \end{tabularx}
\end{table}
    
       


%Intestazione tabella%
\begin{table}[H]
    \centering
    \footnotesize
    \caption{Test Case \#4}
    \begin{tabularx}{\textwidth}{|c|X|}
        \hline
        Applicativo & Mobile\\
        \hline
        Nome & Utente ricerca una struttura  \\
        \hline
        Descrizione & Test che assicura il corretto funzionamento della ricerca di una struttura\\
        \hline
        Note &  \\
        \hline
        Stato & Passato/Fallito\\
        \hline

    \end{tabularx}
    Continua alla pagina seguente
    \setlength{\tabcolsep}{8pt}
    \renewcommand{\arraystretch}{1.5}
\end{table}
%Step%
\begin{table}[H]
    \footnotesize
    \begin{tabularx}{\textwidth}{|c|X|X|X|}
        \hline
        Step\# & Input & Risultato Atteso & Risultato Ottenuto \\
        \hline
         1 & Viene aperta l'homepage dell'applicazione '
         & Viene visualizzato una schermata contenente una mappa con tutte le strutture esistenti in database
         &DA INSERIRE (Come ci si aspettava)\\
          \hline
        2 & Ci si sposta sulla mappa fino a visualizzare la struttura desiderata e ci si clicca sopra
        & Viene visualizzata la schermata della struttura
        & DA INSERIRE\\
         \hline 
        3 & Viene cliccato il tasto back in alto a sinistra
         & Viene mostrata l'homepage & DA INSERIRE\\
          \hline
        4 & Viene cliccato il tasto filtri in alto a destra
         & Viene visualizzata la schermata filtri
         & DA INSERIRE\\
          \hline 
        5 & Vengono inseriti i filtri: "Struttura", "Pompei", "Pub" e viene premuto il tasto "Cerca"
         & Viene visualizzata la lista delle strutture corrispondenti ai criteri di ricerca
         & DA INSERIRE\\
          \hline 
        6 & Viene cliccato il tasto back in alto a sinistra
          & Viene mostrata l'homepage
          & DA INSERIRE\\
           \hline      
        7 & Viene cliccato il tasto filtri in alto a destra
        & Viene visualizzata la schermata filtri
        & DA INSERIRE\\
        \hline       
        
        8 & Viene cliccato il tasto prossimità, viene inserito "100" nel textfield "Distanza massima" viene cliccato il tasto "Cerca"
        & Viene visualizzata la schermata contenete la lista delle strutture entro 100km di prossimità
        & DA INSERIRE\\
        \hline 
        
    \end{tabularx}
\end{table}
    
       

\section{Codice jUnit per unit testing}



\clearpage
% ===========================================================================
%
%		FEDERICO II THESIS TEMPLATE - ENGLISH
%  					* an example of Conclusions
%	 
% 		AUTHOR:  		Antonio Esposito (antonio.esposito103@studenti.unina.it)
%		LAST UPDATED:	2017/06/20
%
% ===========================================================================

\chapter*{Conclusions}
\addcontentsline{toc}{chapter}{Conclusions}

The conclusion of the thesis has to sum up the main considerations and results of the whole work, eventually addressing future steps to continue the work on the discussed topic.
\printglossaries

\end{document}

